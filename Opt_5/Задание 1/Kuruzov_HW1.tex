 \documentclass[12pt]{article}
\usepackage[T2A]{fontenc}
\usepackage[utf8]{inputenc}       

\usepackage[english,russian]{babel}
\usepackage{amsmath,amsfonts,amsthm,amssymb,amsbsy,amstext,amscd,amsxtra,multicol}
\usepackage{verbatim}
\usepackage{tikz}
\usetikzlibrary{automata,positioning}
\usepackage{multicol}
\usepackage{graphicx}
\usepackage[colorlinks,urlcolor=blue]{hyperref}
\usepackage[stable]{footmisc}
\usepackage{ dsfont }

\usepackage{xparse}
\usepackage{ifthen}
\usepackage{bm}
\usepackage{color}

\usepackage{algorithm}
\usepackage{algpseudocode}

\DeclareMathOperator*{\argmin}{argmin}
\DeclareMathOperator*{\argmax}{argmax}
\DeclareMathOperator*{\sign}{sign}
\newcommand\norm[1]{\left\lVert#1\right\rVert}
\begin{document}

\begin{center}
	{Курузов Илья, 678}

	{Задание 1}
\end{center}

\begin{center}
	\textbf{1.}
\end{center}

Далее используются обозначения $\rho_i$ - приблизительное расстояние от путешественника до $i$-ого объекта,  $a_i$ - радиус-вектор $i$-ого объекта, $x$ - искомый радиус-вектор путешественника

Оптимизационная задача: 
$$\argmin_x\sum\limits_{i=1}^{m}|\rho_i-\norm{x-a_i}_2|,$$
$$x \in \mathbb{R}^3.$$

Решением будет точка в $\mathbb{R}^3$, расстояния до которой от объектов "близки" к заданным. Для $i$-ого объекта степенью такой близости будет являться разница $|\rho_i-\norm{x-a_i}_2|$. Для того чтобы учесть разницу между всеми расстояниями, была выбрана целевая функция равная сумме этих разностей(т. е. $l_1$ норму вектора разностей).

\begin{center}
	\textbf{2.}
\end{center}

Характеристикой поселения будет вектор $x$, в котором $x_i$ - номер комнаты $i$-ого студента. В этой задаче рассмотрим функцию выгоды от данного поселения. Она вычисляется следующим образом: изначально она равна нулю, если студенты $i$ и $j$ живет в одной комнате, т.е. $x_i=x_j$, то она увеличивается на $p_{ij}$, если студент $i$ живет в комнате $k$, то значение функции увеличивается на $b_{ij}$. Тогда выражение ждя функции выгоды:
$$f(x) =\left(\sum\limits_{x_i=x_j}p_{ij}+\sum\limits_{x_i=j}b_{ij}\right)$$
и наша задача ее макисимизировать. Тогда как целевую функцию будем использовать $-f(x)$ и минизировать её.

При построении считается, что удовлетворить желание жить вместе каких-то двух студентов в данной системе так же важно, как и удовлетворить желание жить в определнной комнате. В случае если не так, то предлагается использовать поправочные коэффициенты для $b_{ij}$ и $p_{ij}$, зависящие от уточнений к задаче. 

В случае если нет некоторых значений $p_{ij}$, предлагается считать данные значения равными нулю. Т.е. если их поселить вместе, поселяющий ничего не выигрывает.

Оптимизационная задача:
$$\argmin_x\left(-\sum\limits_{x_i=x_j}p_{ij}-\sum\limits_{x_i=j}b_{ij}\right), \eqno{(1)}$$
$$x \in \mathbb{N}^n,\eqno{(2)}$$
$$\forall i = \overline{1,n}\,x_i \leq m,\eqno{(3)}$$
$$\max_j\sum\limits_{x_i=j}1\leq 3\eqno{(4)}$$

Ограничения 2 и 3 связаны с тем, что номера комнат, которыми и являются элементы вектора $x$, есть натуральные числа, не превышающие $m$. Ограничение 4 есть следствие того, что в одной комнате не может жить больше трех человек.

\begin{center}
	\textbf{3.}
\end{center}
Далее количество магазином переобозначено как $M$(вместо $m$).

Рассмотрим вектор $x \in \mathbb{R}^{MN}$, в котором на месте координаты $(n-1)M + k$ стоит количество единиц товара, доставленное с $n$-ого склада в $k$-ый магазин. Для описания целевой функции будем использовать ещё два вектора из $\mathbb{R}^{MN}$: вектор $c$, в котором на месте $(n-1)M + k$ стоит $c_{nk}$, и $t$, в котором на месте $(n-1)M + k$ стоит $t_{nk}$.

Функция денежных расходов определяется, как $\sum\limits_{i=1}^{MN}x_ic_i=(x,c)$. Аналогично, функция временных расходов определяется как $(x,t)$. Считая, что потери денег и времени в данных единицах одинаково плохо, будем использовать как целевую функцию следующее скалярное произведение: $(x, c+t)$.

Оптимизационная задача:
$$\argmin_x(x, c+t), \eqno{(1)}$$
$$x \in \mathbb{R}^{MN},\eqno{(2)}$$
$$\forall n =\overline{1, N}\sum\limits_{i=1}^Mx_{(n-1)M+i}\leq a_i,\eqno{(3)}$$
$$\forall m = \overline{1,M} \sum\limits_{i=1}^Nx_{(i-1)M+m}=b_i. \eqno{(4)}$$

Ограничение 3 появилось из-за ограниченности товара на складе, ограничение 4 есть выражение необходимости доставить всем магазинам требуемое количество товара.

\begin{center}
	\textbf{4.}
\end{center}

I.Нормой вектора называется функция $\norm{.}:\mathbb{R}^n\rightarrow \mathbb{R}$, удовлетворяющая следующим условиям:

1. $\norm{x} = 0 \leftrightarrow x = 0$.

2. $\norm{x} \geq 0$

3. $\norm{x+y} \leq \norm{x} + \norm{y}$

4. $\forall \alpha \in \mathbb{R}\,\norm{\alpha x}=|\alpha|\norm{x}$

Две нормы $p(x)$ и $q(x)$ называются эквивалентными, если

$$\exists C_1, C_2\neq 0,\,\forall x \in \mathbb{R}^n\,C_1p(x)\leq q(x)\leq C_2p(x)$$.

Доказательство эквивалентности $l_1$ и $l_{\infty}$ норм:

$$\norm{x}_{\infty} = \max_i{|x_i|}\leq \sum\limits_{i=1}^{n}|x_i|=\norm{x}_1$$ 
$$n\norm{x}_{\infty} = n\max_i{|x_i|}\geq \sum\limits_{i=1}^{n}|x_i|=\norm{x}_1$$ 
$$\norm{x}_{\infty}\leq\norm{x}_1\leq n\norm{x}_{\infty}$$

Доказательство эквивалентности $l_{\infty}$ и $l_2$ норм:

$$\norm{x}_2 = \sqrt{\sum\limits_{i=1}^nx_i^2}\leq \sqrt{n\max_i{|x_i|^2}}=\sqrt{n}\norm{x}_{\infty}$$
$$\norm{x}_2 = \sqrt{\sum\limits_{i=1}^nx_i^2}\geq \sqrt{\max_i{|x_i|^2}}=\norm{x}_{\infty}$$
$$\norm{x}_{\infty}\leq\norm{x}_1\leq \sqrt{n}\norm{x}_{\infty}$$

Эквивалентность $l_{1}$ и $l_2$ норм следует из доказанного и из того, что эквивалентность норм - это отношение эквивалентности.

II. Норма матрицы $\norm{A}$ называется порожденной векторной нормой $\norm{x}$, если она определена как $$\norm{A}=\max_{\norm{x}=1}\norm{Ax}$$.

$L_{\infty}$ норма:

$$\norm{A} = \max_{\norm{x}=1}max_{i}\left|\sum\limits_{j=1}^na_{ij}x_j\right|\leq \max_j|x_j|\max_i\sum\limits_{j=1}^n|a_{ij}|=\max_i\sum\limits_{j=1}^n|a_{ij}|$$

Данная оценка достигается при $x =\{x_i = \sign(a_{ji})\}$, где $j$ - номер строки, на которой достигается максимум $\sum\limits_{j=1}^n|a_{ij}|$.

$$\norm{A} =\max_i\sum\limits_{j=1}^n|a_{ij}|$$

$L_{1}$ норма:

$$\norm{A} = \max_{\norm{x}=1} \norm{Ax}_1= \max_{\norm{x}=1} \sum\limits_{i=1}^n\left|\sum\limits_{j=1}^na_{ij}x_j\right|\leq \max_{\norm{x}=1} \sum\limits_{i=1}^n\sum\limits_{j=1}^n|a_{ij}x_j|\\ \leq \max_{\norm{x}=1} \sum\limits_{i=1}^n\left(\max_j|a_{ij}|\sum\limits_{j=1}^n|x_j|\right)$$

$$\norm{A} \leq \max_i\sum\limits_{j=1}^n|a_{ij}|$$

Это оценка достигается, если взять такой вектор $x$, что на $i$-ой позиции стоит 1, а на остальных нули, где $i$ - номер, на котором достигается максимум $\sum\limits_{j=1}^n|a_{ij}|$.
$$\norm{A} =\max_i\sum\limits_{j=1}^n|a_{ij}|$$

Норма Фробениуса(аналог векторной $l_2$ нормы):

$$\norm{A} = \left(\sum\limits_{i,j=1}^na_{ij}^2\right)^{1/2}$$

\begin{center}
	\textbf{5.}
\end{center}

I. Пусть $x,\,y$ - элементы линейного простанства $L$ со скалярным произведением $(x,y)$ и с нормой $x = \sqrt{(x,x)}$, тогда верно неравенство Коши-Буняковского-Шварца:

$$|(x,y)|\leq \norm{x}\norm{y}$$

Доказательство:

$$(\alpha x -y, \alpha x - y) = \alpha^2 \norm{x}^2-2\alpha(x,y)+\norm{y}^2\geq 0$$

В случае, если $x=0$, неравенство верно. Иначе положим $\alpha = \frac{(x,y)}{\norm{x}^2}$. Тогда получим неравенство, эквивалентное неравенству Коши-Буняковского-Шварца. $\square$

II. SVD разложение матрицы $A$ - представление матрицы в виде произведения трех матриц:

$$A = U\Sigma V^T,$$
где $U$ и $V$ - ортогональные матрицы состоящие из левых и правых сингулярных векторов, $\Sigma$ - матрица, на диагонали которой стоят соответствующие сингулярные числа.

QR разложение матрицы $A$ - представление матрицы в виде произведения двух матриц:

$$A = QR,$$
где $Q$ - ортогольная матрица, $R$ - треугольная матрица.

Для решения данной системы $Ax=b$ можно использовать SVD разложение для матрицы $A$ и умножать обе части на обратные матрицы для матриц в умножении (для ортогональной и диагональной нахождение обратной занимает $O(n^2)$). И далее сложность зависит от способа умножения матриц(к примеру, при помощи алогоритма Штрассена сложность $O(n^{\log_27})$). Таким образом, без учета сложности вычислений SVD разложения, решение линейных систем данным методом быстрее, чем методом Гаусса(кубическое время).

При применении QR разложения умножим обе части на $Q^T$(умножение матрицы на вектор и вычисление транспонированной матрицы - $O(n^2)$) и используем метод Гаусса для $R$(поскольку матрица треугольная - $O(n^2)$). Значит, использование
QR разложения, без учета времени на нахождение этого разложения, требует $O(n^2)$.

III. Разреженная матрица - матрица, большАя часть элементов которой равны нулю. Точная граница между разреженными и неразреженными матрицами отсутствует.

Способ хранения: хранение по строкам пар (значение элемента, индекс стобца). Способ нахождения элемента с данным индексом может осущетвляться с помощью бинарного поиска по строке. Аналогично можно хранить по столбцам пар.

IV. Доказательство ассоциативности матричного умножения. Далее используется обозначение $[A]_{ij}$, $a_{ij}$ - элемент матрицы $A$ с индексами $i$  и $j$. Рассмотрим три матрицы $A_{m, n}, B_{n, k}, C_{k. l}$ и два их произведения $(AB)C$ и $A(BC)$.

$$[(AB)C]_{ij} = \sum\limits_{p=1}^k[AB]_{ip}c_{pj}= \sum\limits_{p=1}^k\left(\sum\limits_{d=1}^na_{id}b_{dp}\right)c_{pj} = \sum\limits_{d=1}^na_{id}\sum\limits_{p=1}^kb_{dp}c_{pj} =$$
$$=\sum\limits_{d=1}^na_{id}[BC]_{dj}=[A(BC)]_{ij}$$

Из выше написанного покомпонентного равенства следует ассоциативность матричного умножения.
\end{document}