 \documentclass[12pt]{article}
\usepackage[T2A]{fontenc}
\usepackage[utf8]{inputenc}       

\usepackage[english,russian]{babel}
\usepackage{amsmath,amsfonts,amsthm,amssymb,amsbsy,amstext,amscd,amsxtra,multicol}
\usepackage{verbatim}
\usepackage{tikz}
\usetikzlibrary{automata,positioning}
\usepackage{multicol}
\usepackage{graphicx}
\usepackage[colorlinks,urlcolor=blue]{hyperref}
\usepackage[stable]{footmisc}
\usepackage{ dsfont }

\usepackage{xparse}
\usepackage{ifthen}
\usepackage{bm}
\usepackage{color}

\usepackage{algorithm}
\usepackage{algpseudocode}

\DeclareMathOperator*{\argmin}{argmin}
\DeclareMathOperator*{\argmax}{argmax}
\DeclareMathOperator*{\sign}{sign}
\newcommand\norm[1]{\left\lVert#1\right\rVert}
\begin{document}

\begin{center}
	{Курузов Илья, 678}

	{Задание 2}
\end{center}

\begin{center}
	\textbf{1.}
\end{center}

Далее индекс нормы отброшен, в силу того что используется только $l2$ норма. 

I. Рассмотрим две точки $x_0$ и $x_i$. Далее отождествляется понятие точки в пространстве и соответствующего ей радиус-вектора. Докажу эквивалентность следующих условий:
$$\norm{x_0-x} \leq \norm{x_i - x}\eqno{(1)}$$
$$(x, x_i - x_0) \leq \frac{1}{2}\left(\norm{x_i}^2-\norm{x_0}^2\right)\eqno{(2)}$$

Из неотрицательности нормы и неравенства (1) следует:

$$2(x,x_i-x_0)\leq\norm{x_i}^2-\norm{x_0}^2$$

Деля обе части на 2, получаю неравенство (2). Из выше написанного следует, что совокупность всех ограничений $(1)$ равносильна следующему условию:

$$Ax \preceq b,$$ 

где $a_{ij} = [x_i-x_0]_j$, $b_i = \frac{1}{2}\left(\norm{x_i}^2-\norm{x_0}^2\right)$. Значит, область Вороного действительно многоугольник.$\square$

II. Докажем обратное. Пусть $a_i$ - $i$-ая строка матрицы. Выберем точку $x_0$, удовлетворяющую условию 
$$Ax \preceq b\eqno{(1)}$$
$$(a_i,x_0) = b_i - \frac{1}{2}\norm{a_i}^2, \forall i = \overline{1,n}\eqno{(2)}$$
и рассмотрим точку $x_i = x_0 + a_i$. Немного о существование $x_0$, удовлтворяющей условиям $(1)$ и $(2)$. Она существует, если условие 1 не задает пустого множества. Поскольку совокупность условий $(2)$ представима в виде СЛАУ с непротиворечивыми уравнениями(это следует из того, что условие 1 не задает пустого множества), это действительно так. 

Теперь докажу эквивалентность следующих неравенств:
$$\norm{x_0-x} \leq \norm{x_i - x}\eqno{(3)}$$
$$(x, a_i) \leq b_i\eqno{(4)}$$

Проведя преобразования неравенства $(1)$, получаю аналогичное уравнение:

$$(x, x_i - x_0) \leq \frac{1}{2}\left(\norm{x_i}^2-\norm{x_0}^2\right)$$

С учетом следующего равенства(следует из построения $x_i$)
$$\norm{x_i}^2 - \norm{x_0}^2 = 2(a_i,x_0)+\norm{a_i}^2,$$
условия $(2)$ для $x_0$ и принятых обозначений, получаю равенство $4$. Значит, любой многоугольник определяет область Вороного. $\square$

III. Обозначим $V(x_0, Y)= \{x\in\mathbb{R}^n|\norm{x_0-x} \leq \norm{y - x},\forall y \in Y\}$, где $x_0$ - точка из $\mathbb{R}^n$, $Y$ - конечное множество точек из $\mathbb{R}^n$.

Разбиением Вороного для точек из множества $X = \{x_0, x_1 ... x_k\}$ назовем следущее разбиение $\mathbb{R}^n$:

$$S = \{V(x_1, X\backslash x_1)\ldots V(x_k, X\backslash x_k)\}.$$

Пусть есть $k$ многоугольник:

$$A_mx \preceq b_m,$$

в совокупности которые образуют разбиение $\mathbb{R}^n$ на многоугольники.

Найдем множество $X_j$ решений системы из следующих уравнений:

$$(a_i^j,x_j) = b_i^j - \frac{1}{2}\norm{a_i^j}^2, \forall i in \overline{1,n},$$
где $a_i^j$ - $i$-aя строка $m$-ой матрицы.

И для каждого множества решений $X_j$ найдем соответствующие множества с остальными индексами. $$\mathcal{X}_j = \cup_{x\in X_j}( x_1\ldots x_k),$$

где

\begin{equation*}
x_i = 
 \begin{cases}
   x+a_i &\text{if  $i<j$}\\
   x &\text{if $i=j$}\\
   x+a_{i-1} &\text{if  $i>j$}
 \end{cases}
\end{equation*}


Таким образом $\mathcal{X}_j$ - всевозможные множества точек задающих область Вороного для $x_j$, совпадающие с $j$-ым многоугольником. Это следует из рассуждений представленных во втором пункте и из того, что контрукции без учета сложностей, связанных с нумерацией, в этом пункте и предыдущем одинаковы.

Тогда любой элемент $\cap_{i=1}^k\mathcal{X}_i$ задает такое множество точек, что если построить область Вороного на этих точках для $x_i$ она совпадет с $i$-ым многоугогольником для всех $i$. Что собственно и требовалось построить.

\begin{center}
	\textbf{2.}
\end{center}

I. Матрица $A$  положительна полуопределена. Пусть $x_1,\,x_2\in C$, тогда для того, чтобы множество $C$ было выпукло, необходимо и достаточно, $x = \theta x_1+(1-\theta)x_2\in C,\forall \theta \in [0,1]$.

$$x^TAx+b^Tx+c=$$
$$\theta(x_1^TAx_1+b^Tx_1+c) +(1-\theta)(x_2^TAx_2+b^Tx_2+c) + x^TAx - \theta x_1^TAx_1 - (1-\theta)x_2^TAx_2 \leq$$
$$\leq x^TAx - \theta x_1^TAx_1 - (1-\theta)x_2^TAx_2=$$
$$=-\theta(1-\theta)(x_1^TAx_1-x_1^TAx_2-x_2^TAx_1+x_2^TAx_2)=$$
$$=-\theta(1-\theta)(x_1-x_2)^TA(x_1-x_2)$$


Из положительной полуопределенности матрицы $A$ и отрицательности коэффициента перед ней следует неположительность правой части, а, значит, $x \in C$. На этом доказательство выпуклости множества $C$ можно считать завершенным.$\square$

\begin{center}
	\textbf{3.}
\end{center}

I. Далее $x_1,\,x_2$ - элементы множества $A$(различных для каждого пункта), $x = \theta x_1+(1-\theta)x_2$ в доказательствах на выпуклость и афинность и $x = \theta x_1$ в доказательстве на конусность. Когда идет доказательство на выпуклость $\theta \in [0,1]$, на афинность - $\theta \in \mathbb{R}$, на конусность - $\theta \geq 0$.

1) $A = \{x\in \mathbb{R}^n|\alpha\leq a^Tx\leq \beta\}$

Данное множество выпукло, не афинное(при $\alpha \neq \beta$), не конус(при $\alpha \neq \beta$). Если $\alpha = \beta$, то множество $A$ - гиперплоскость, а в курсе доказано, что она выпукла, афинна и является конусом.

\textbf{Выпуклость:}

$$a^Tx = \theta a^Tx_1 + (1-\theta)a^Tx_2x$$

$$a^Tx \geq \theta \alpha + (1 - \theta)\alpha = \alpha$$

$$a^Tx \leq \theta \beta + (1-\theta)\beta = \beta$$

$$x\in A$$

Далее идут рассуждения, в предположении что $A$ не пусто.

\textbf{Афинность:}

Пусть $x_1$ - точка из плоскости $a^Tx=\beta$, $x_2$ - точка из плоскости $a^Tx=\alpha$, $\theta =2$
$$x = 2x_1 - x_2$$

$$a^Tx = 2\beta - \alpha>\beta$$

$$x \notin A$$


\textbf{Конус:}

Пусть $x_1$ - точка из плоскости $a^Tx=\beta$, $\theta =2$.
$$x = 2x_1$$

$$a^Tx = 2\beta>\beta$$

$$x \notin A$$

2) $A=\{x\in\mathbb{R}^n|a_1^Tx\leq b_1,\,a_2^Tx\leq b_2\}$

\textbf{Выпуклость:}

Множество выпукло.

$$a_1^Tx = \theta a_1^Tx_1 + (1-\theta)a_1^Tx_2x\leq b_1$$
$$a_2^Tx = \theta a_2^Tx_1 + (1-\theta)a_2^Tx_2x\leq b_2$$
$$x\in A$$

\textbf{Афинность:}

Множество не афинно, если не задает гиперплоскость, т. е. если не выполнены одновременно условия $a_1=-ka_2$ и $b_1 = -kb_2$, $k>0$.

Пусть $x_1$ - точка из пересечения плоскостей $a_i^Tx=b_i$, $i = 1, 2$ со множеством $A$(пересечение не пусто, следует из того, что граница пересечения принадлежит объединению границ пересекаемых множеств и непустоты $A$), $x_2$ - произвольная точка из внутренности $A$. Без ограниений общности, будем считать, что $x_1\in\{x|a_1^Tx= b_1\}$. $\theta_1 =2$.
$$x = 2x_1 - x_2$$
$$a_1^Tx = 2b_1- q, q<b_1$$
$$a_1^Tx> b_1$$
$$x\notin A$$

\textbf{Конус:}

Множество не конус, если $b_1 = 0,\,b_2=0$.
Рассмотрим все $x_0 \in A$, для которого $a_1^Tx_0 > 0$ или $a_2^Tx_0 > 0$. И

Пусть выше описанный $x_0$ существует. $x_1$ - точка, описанная выше. Без ограниений общности, будем считать, что $x_1\in\{x|a_1^Tx>0\}$. $\theta_1 =\max{(1, \frac{2b_1}{a_1^Tx})}$.
$$x = \theta_1x_1$$
$$a_1^Tx = \max{(a_1^Tx_1, 2b_1)} > b_1, q<b_1$$
$$x\notin A$$

Условия $a_1^Tx_0 > 0$ или $a_2^Tx_0 > 0$ не выполнены, а, значит, $b_i\leq 0$, $i=1,2$.

Если выше описанный $x_0$ не существует, докажу, что множество не конус. Пусть $x_1$ - точка из пересечения плоскостей $a_i^Tx=b_i$, $i = 1, 2$, $b_i<0$ со множеством $A$(пересечение не пусто, следует из того, что граница пересечения принадлежит объединению границ пересекаемых множеств и непустоты $A$), будем считать, что $x_1\in\{x|a_1^Tx= b_1\}$. $\theta_1 =\frac{1}{2})$.

$$a_1^Tx=\frac{1}{2}a_1^Tx_1=\frac{1}{2}b_1> b_1$$

$$x\notin A$$

Если $b_1 = 0,\,b_2=0$, то множество конус, причем выпуклый.

$$a_1^Tx = \theta_1 a_1^Tx_1 + \theta_2 a_1^Tx_2 \leq 0$$
$$a_2^Tx = \theta_1 a_2^Tx_1 + \theta_2 a_2^Tx_2 \leq 0$$
$$x\in A$$

3) $$A=\{x\in\mathbb{R}^n|\norm{x-x_0}_2\leq\norm{x-y}_2, \forall y\in S\subseteq \mathbb{R}^n\}$$

\textbf{Выпуклость:}

Множество выпукло.

$$\norm{x-x_0}\leq \theta\norm{x_1-x_0}+(1-\theta)\norm{x_2-x_0}\leq \norm{x-y}$$
$$x\in A$$

\textbf{Афинность:}

Множество не афинно, нет такого $y \in S$, что $x_0=y$. Иначе $A=\{x_0\}$ и это множество очевидно афинно.

Пусть $y =\argmin\limits_S\norm{y-x_0}$(предполагается, что $S$ не пусто, иначе множество $A$ не определено).

Пусть $x_1 = x_0$, $x_2 = \frac{y+x_0}{2}$(принадлежность этих точек проверяется тривиальной подстановкой), $\theta = -\frac{1}{3}$.

$$x=\frac{2y+x_0}{3}$$

$$\norm{x-x_0} = \frac{2}{3}\norm{y - x_0}> \frac{1}{3}\norm{y - x_0} = \norm{y-x}$$

$$x\notin A$$

\textbf{Конус:}

Множество может быть как конусом, так и не конусом.

Пример конуса: $n=2,\,x_0=(0,-1), \,S = {(0,1)}$. Тогда множество $A$ - полуплоскость под прямой $y=0$. Если взять произвольные $x_1 \in A, \theta\geq 0$, то получим, что координата по $y$ остается отрицательной, а, значит, $\theta x_1 \in A$.

Пример не конуса: $n=2,\,x_0=(0,0), \,S = {(0,2)}$. Тогда множество $A$ - полуплоскость под прямой $y=1$. Если взять $x_1=(0,1),\theta=2$, то получим $\theta x_1\notin A$.

4) Далее $\theta$ из условия переобозначена, как $\alpha$ во избежание путаницы. $$A=\{x\in\mathbb{R}^n|\norm{x-a}_2\leq \alpha\norm{x-b}_2\}$$

\textbf{Выпуклость:}

Множество выпукло.

$$\norm{x-a}\leq \theta\norm{x_1-a}+(1-\theta)\norm{x_2-a}\leq \alpha\norm{x-b}$$
$$x\in A$$

\textbf{Афинность:}

Множество не афинно.

Пусть $x_1 = a$, $x_2$ - произвольная точка из множества точек, удовлетворяющих условию $\norm{x-a}_2= \alpha\norm{x-b}_2$, $\theta = 2$.

$$x= 2x_2-x_1$$

$$\norm{x-a} = 2\norm{x_2 -a} = 2\alpha\norm{x-b}>\alpha\norm{x-b}$$
$$x\notin A$$

\textbf{Конус:}

Множество может быть как конусом, так и не конусом.

Пример конуса: $n=2,\,a=(0,-1), \,b = (0,1), \alpha=1$. Тогда множество $A$ - полуплоскость под прямой $y=0$. Если взять произвольные $x_1 \in A, \theta\geq 0$, то получим, что координата по $y$ остается отрицательной, а, значит, $\theta x_1 \in A$.

Пример не конуса: $n=2,\,a=(0,0), \,b = (0,2), \alpha=1$. Тогда множество $A$ - полуплоскость под прямой $y=1$. Если взять $x_1=(0,1),\theta=2$, то получим $\theta x_1\notin A$.

II. Докажу, что пересечение любого (конечного или бесконечного) числа выпуклых конусов является выпуклым конусом.

Пусть $\mathcal{F}$ - семейство выпуклых конусов. Пусть $C = \cap_{X\in\mathcal{F}}X$. Докажем, что $С$ - выпуклый конус.

Пусть $x_1,x_2\in C$. Значит, $x_1,x_2\in X,\forall X\in \mathcal{F}$. Поскольку любой $X\in \mathcal{F}$ - выпуклый конус, то $$\theta_1 x_1+\theta_2 x_2\in X,\forall X\in \mathcal{F}, \theta_1,\theta_2\geq 0.$$

Значит, $$\theta_1 x_1+\theta_2 x_2\in X,\forall x_1,x_2\in C, \theta_1,\theta_2\geq 0.$$

Что и означает, что $C$ - выпуклый конус.$\square$

Докажу, что пересечение любого (конечного или бесконечного) числа аффинных множеств является аффинным множеством.

Пусть $\mathcal{F}$ - семейство аффинных множеств. Пусть $А = \cap_{X\in\mathcal{F}}X$. Докажем, что $С$ - аффинное множество.

Пусть $x_1,x_2\in A$. Значит, $x_1,x_2\in X,\forall X\in \mathcal{F}$. Поскольку любой $X\in \mathcal{F}$ - аффинное множество, то $$\theta x_1+(1-\theta) x_2\in X,\forall X\in \mathcal{F}, \theta \in \mathbb{R}.$$

Значит, $$\theta x_1+(1-\theta) x_2\in X,\forall x_1,x_2\in A, \theta \in \mathbb{R}.$$

Что и означает, что $A$ - аффинное множество.$\square$
\end{document}