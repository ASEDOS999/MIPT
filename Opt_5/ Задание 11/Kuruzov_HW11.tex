 \documentclass[12pt]{article}
\usepackage[T2A]{fontenc}
\usepackage[utf8]{inputenc}       

\usepackage[english,russian]{babel}
\usepackage{amsmath,amsfonts,amsthm,amssymb,amsbsy,amstext,amscd,amsxtra,multicol}
\usepackage{verbatim}
\usepackage{tikz}
\usepackage{pgfplots}
\pgfplotsset{compat=1.9}
\usetikzlibrary{automata,positioning}
\usepackage{multicol}
\usepackage{graphicx}
\usepackage[colorlinks,urlcolor=blue]{hyperref}
\usepackage[stable]{footmisc}
\usepackage{ dsfont }

\usepackage{xparse}
\usepackage{ifthen}
\usepackage{bm}
\usepackage{color}

\usepackage{algorithm}
\usepackage{algpseudocode}

\DeclareMathOperator*{\argmin}{argmin}
\DeclareMathOperator*{\argmax}{argmax}
\DeclareMathOperator*{\sign}{sign}
\DeclareMathOperator*{\aff}{aff}
\DeclareMathOperator*{\conv}{conv}
\DeclareMathOperator*{\relint}{relint}
\DeclareMathOperator*{\intset}{int}
\DeclareMathOperator*{\dom}{dom}
\DeclareMathOperator*{\tr}{tr}
\DeclareMathOperator*{\st}{s.t. }
\newcommand\norm[1]{\left\lVert#1\right\rVert}
\newcommand\abs[1]{\left|#1\right|}
\begin{document}

\begin{center}
	{Курузов Илья, 678}

	{Задание 11}
\end{center}

\begin{center}
	\textbf{Задача 2}
\end{center}

Решим эквивалентную задачу:

$$\min -x_1-4x_2-x_3$$
$$2x_1+5x_2+x_3=4$$
$$2x_1-5x_2-x_3=0$$
$$x_i \geq 0, \forall i$$

И еще раз приведем к эквивалентному виду:

$$\min -x_1-4x_2-x_3$$
$$5x_2+x_3=2$$
$$x_1 = 1$$
$$x_i \geq 0, \forall i$$

Угловая точка - $(1,0,2)^\top$, $\mathcal{B} = \{1, 3\}$. Заметим, что матрица образуемая первым и третьим столбцом $A$ не вырождена. Приведем ее к единичному виду и запишем таблицу для начальной итерации.

\begin{table}[H]
\caption{Нулевая итерация}
\begin{center}
\begin{tabular}{|l|lll}
\hline
 &{ $x_1$} & {$x_2$} & \multicolumn{1}{l|}{$x_3$}\\ \hline
$-\textbf{c}^\top \textbf{x} = 3$ &          0             &    1                   &\multicolumn{1}{l|}{0}  \\ \hline
$x_1 = 1$ &          0             &    5                   &\multicolumn{1}{l|}{1}  \\
$x_3 = 2$ &           1            &     0                  &\multicolumn{1}{l|}{0}                      \\\hline
\end{tabular}
\end{center}
\end{table}

Значит, $(1,0,2)^\top$ дает минимум.

$$\min -x_1-4x_2-x_3 = (-x_1-4x_2-x_3)|_{\textbf{x}=(1,0,2)^\top} =-3 $$

Или возвращаясь к исходной задаче:

$$\boxed{\max (x_1+4x_2+x_3) = (x_1+4x_2+x_3)|_{\textbf{x}=(1,0,2)^\top} =3}$$

\begin{center}
	\textbf{Задача 3}
\end{center}

Составим вспомогательную задачу. Все $b_i>0$ поэтому никаких дополнительных преобразований не требуется. Начальная угловая точка $\textbf{x}_0 = (0,0,0,0,0,2,2,1)^\top$, $\mathcal{B} = \{6,7,8\}$.

\end{document}

