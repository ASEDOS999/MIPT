 \documentclass[12pt]{article}
\usepackage[T2A]{fontenc}
\usepackage[utf8]{inputenc}       

\usepackage[english,russian]{babel}
\usepackage{amsmath,amsfonts,amsthm,amssymb,amsbsy,amstext,amscd,amsxtra,multicol}
\usepackage{verbatim}
\usepackage{tikz}
\usepackage{pgfplots}
\pgfplotsset{compat=1.9}
\usetikzlibrary{automata,positioning}
\usepackage{multicol}
\usepackage{graphicx}
\usepackage[colorlinks,urlcolor=blue]{hyperref}
\usepackage[stable]{footmisc}
\usepackage{ dsfont }

\usepackage{xparse}
\usepackage{ifthen}
\usepackage{bm}
\usepackage{color}

\usepackage{algorithm}
\usepackage{algpseudocode}

\DeclareMathOperator*{\argmin}{argmin}
\DeclareMathOperator*{\argmax}{argmax}
\DeclareMathOperator*{\sign}{sign}
\DeclareMathOperator*{\aff}{aff}
\DeclareMathOperator*{\conv}{conv}
\DeclareMathOperator*{\relint}{relint}
\DeclareMathOperator*{\intset}{int}
\DeclareMathOperator*{\dom}{dom}
\DeclareMathOperator*{\tr}{tr}
\DeclareMathOperator*{\st}{s.t. }
\newcommand\norm[1]{\left\lVert#1\right\rVert}
\newcommand\abs[1]{\left|#1\right|}
\begin{document}

\begin{center}
	{Курузов Илья, 678}

	{Задание 9}
\end{center}

\begin{center}
	\textbf{Задача 1}
\end{center}

1. Решим оптимизационную задачу:

\begin{gather}
\min\limits_{\textbf{x}\in\mathbb{R}^3}x^2+1\\
\st\,(x-2)(x-4)\leq 0
\end{gather}

Условный субдифференциал:

\begin{equation}
\begin{split}
\partial_X f(x) = \partial f(x) + N(x|X) =\\ 
=2x+1 + \begin{cases}
\alpha, \alpha \leq 0, \text{$x = 2$}\\
\alpha, \alpha \leq 0, \text{if $x = 2$}\\
0, \text{else}
\end{cases}
\end{split}
\end{equation}

Получаем, что $0\in \partial_X f(2)$. В силу выпуклости целевой функции и бюджетного множества, это точка минимума:
$$\boxed{\min f(x) = f(2) = 5}$$

2. 

\begin{tikzpicture}
\begin{axis}[
	xlabel = {\text{Бюджетное множество}},
	ylabel = {Целевая функция},
	minor tick num = 2,
	xmin = 1,
	xmax = 5,
	domain = 2:4
]
\addplot[blue] {x^2 + 1};
\addplot table{
	x y
	2 5
};
\end{axis}
\end{tikzpicture}

3.

\begin{tikzpicture}
\begin{axis}[
	legend pos = north west,
	xlabel = {\text{Бюджетное множество}},
	ylabel = {Лагранжиан},
	minor tick num = 2,
	xmin = 1,
	xmax = 5,
	domain = 2:4
]
\legend{ 
	$p^* =  5$,
	$\lambda = 1$,
	$\lambda = 2$, 
	$\lambda = 5$, 
	$\lambda = 10$, 
};
\addplot[yellow] {5};
\addplot[violet] {x^2 + 1 + (x-2)*(x-4)};
\addplot[blue] {x^2 + 1 + 2*(x-2)*(x-4)};
\addplot[red] {x^2 + 1 + 5*(x-2)*(x-4)};
\addplot[green] {x^2 + 1 + 10*(x-2)*(x-4)};
\end{axis}
\end{tikzpicture}

Как видно из рисунка, $p^* \geq \inf\limits_x L(x,\lambda)$

4. 
\begin{equation}
\begin{split}
g(\lambda) = \inf\limits_{x\in X} (x^2 + 1 + \lambda(x-2)(x-4)) = \\
=  1 + 8\lambda + \inf\limits_{x \in X}((\lambda + 1)x^2 -6\lambda x)=\\
= 1 + 8\lambda + \left((\lambda + 1)x^2 -6\lambda x\right)\Big|_{x = \max\left(2,\frac{3\lambda}{\lambda + 1}\right)}=\\
= \begin{cases}
\frac{-\lambda^2 + 9\lambda + 1}{\lambda+1},\text{if $\lambda \geq 2$},\\
5,  \text{if $0\leq \lambda \leq 2$}
\end{cases} =\\
= \begin{cases}
-\lambda + 10 - \frac{9}{\lambda+1},\text{if $\lambda \geq 2$},\\
5,  \text{if $0\leq \lambda \leq 2$}
\end{cases}
\end{split} 
\end{equation}

Функция всюду $g$ дифференцируема, причем $g'(\lambda) \leq 0, \forall \lambda \geq 0$. Значит, $g(0)$ дает  максимум, т.е. решение двойственной задачи:

$$\boxed{\max_{\lambda\,:\,\lambda\geq 0}g(\lambda) = g(0) = 5}$$

$d^* = p^*$ и поэтому свойство сильной двойственности выполнено.
\end{document}