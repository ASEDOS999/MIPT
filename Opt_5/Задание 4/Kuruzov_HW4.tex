 \documentclass[12pt]{article}
\usepackage[T2A]{fontenc}
\usepackage[utf8]{inputenc}       

\usepackage[english,russian]{babel}
\usepackage{amsmath,amsfonts,amsthm,amssymb,amsbsy,amstext,amscd,amsxtra,multicol}
\usepackage{verbatim}
\usepackage{tikz}
\usetikzlibrary{automata,positioning}
\usepackage{multicol}
\usepackage{graphicx}
\usepackage[colorlinks,urlcolor=blue]{hyperref}
\usepackage[stable]{footmisc}
\usepackage{ dsfont }

\usepackage{xparse}
\usepackage{ifthen}
\usepackage{bm}
\usepackage{color}

\usepackage{algorithm}
\usepackage{algpseudocode}

\DeclareMathOperator*{\argmin}{argmin}
\DeclareMathOperator*{\argmax}{argmax}
\DeclareMathOperator*{\sign}{sign}
\DeclareMathOperator*{\relint}{relint}
\DeclareMathOperator*{\dom}{dom}

\newcommand\norm[1]{\left\lVert#1\right\rVert}
\newcommand\abs[1]{\left|#1\right|}
\begin{document}

\begin{center}
	{Курузов Илья, 678}

	{Задание 4}
\end{center}

\begin{center}
	\textbf{Задача 1.}
\end{center}

\begin{center}
	\textbf{10.}
\end{center}

1) $$f(x) =\abs{\abs{x}-1}$$

Данная фунция не является выпуклой. Возьмем $x_1 = -1{,}5,x_2=1{,}5,\alpha =0{,}5$ и получаем:

$$f( \alpha x_1 + (1-\alpha)x_2) = f(0) = 1 \geq 0{,}5 = \alpha f(x_1)+(1-\alpha)f(x_2)$$

2) $$f(x,y) = x^2y^2$$

Функция $f$ определена на $\mathbb{R}^2$, а потому $\relint(X_f) = \mathbb{R}^2$.Гессиан $f$:

$$H(f) = \begin{pmatrix} 2y^2 & 4xy \\ 4xy & 2x^2 \end{pmatrix}$$

$$\Delta_1 = 2y^2 \geq 0, \forall x,y$$
$$\Delta_2 = (2y^2)(2x_2)-(4xy)(4xy)=-12(xy)^2 < 0, \forall x,y \neq 0$$

Из дифференциального критерия второго порядка следует, что $f(x,y)$ не является выпуклой.

\begin{center}
	\textbf{11.}
\end{center}

Рассмотрим следующую функцию:
$$f(x) = -\sin(x), X = [0, \pi]$$
Тогда
$$f^2(x) = \frac{1-\cos(2x)}{2}, X = [0, \pi]$$

Для этих функций $\relint(X) = (0, \pi)$. Найдем вторые производные, которые собственно и являются единственной компонентой гессиана, наших функций:

$$f''(x) = \sin(x) \geq 0, \forall x \in \relint(X)$$
$$(f^2(x))''= 2\cos(2x) < 0, x = \frac{\pi}{2} \in \relint(X)$$

Значит, мы привели пример выпуклой $f$, для которой $f^2$ не выпукла.

\begin{center}
	\textbf{11.}
\end{center}

$$f(x) = -\sin(x), X_f = [0, \pi]$$
$$g(x) = x^2, X_g = [-1, 0]$$

Функия $g$ выпукла, поскольку
$$g''(x) = 2 \geq 0, \forall x \in \relint(X_g)$$
Выпуклость функции $f$ доказана в предыдущем пункте. 

$$g(f(x)) = \frac{1-\cos(2x)}{2}, X = [0, \pi]$$

Функция $g \circ f$ не является выпуклой, что доказано в предыдущем пункте.

\begin{center}
	\textbf{Задача 2.}
\end{center}

Докажем обратное неравенство Йенсена по индукции.

I. База индукции при $n=2$. Докажем ее от противного.

$$f(\lambda \textbf{x}_1 + (1-\lambda)\textbf{x}_1) < \lambda f(\textbf{x}_1) + (1-\lambda)f(\textbf{x}_1), \forall \lambda > 0$$

Зафиксируем $\textbf{x}_1,\textbf{x}_2$ из $\dom f$ и рассмотрим $\lambda = \frac{3}{2}$.

Тогда из нашего предположения и  из условия о выпуклости функции получаем два неравенства:

$$f\left(\frac{3}{2} \textbf{x}_1 - \frac{1}{2}\textbf{x}_1\right) < \frac{3}{2} f(\textbf{x}_1) - \frac{1}{2}f(\textbf{x}_1)$$
$$f\left(\frac{1}{2} \textbf{x}_1 + \frac{1}{2}\textbf{x}_1\right) \leq \frac{1}{2} f(\textbf{x}_1) + \frac{1}{2}f(\textbf{x}_1)$$

Далее пользуясь этими неравенствами и выпуклостью функции, получаем следующее:

$$f(\textbf{x}_1) = $$
$$=f\left(\frac{1}{2}\left(\frac{3}{2} \textbf{x}_1 - \frac{1}{2}\textbf{x}_1\right)+\frac{1}{2}\left(\frac{1}{2} \textbf{x}_1 + \frac{1}{2}\textbf{x}_1\right)\right)\leq$$
$$\leq \frac{1}{2}f\left(\frac{3}{2} \textbf{x}_1 - \frac{1}{2}\textbf{x}_1\right)+\frac{1}{2}f\left(\frac{1}{2} \textbf{x}_1 + \frac{1}{2}\textbf{x}_1\right) <$$

$$<\frac{3}{4} f(\textbf{x}_1) - \frac{1}{4}f(\textbf{x}_1) + \frac{1}{4} f(\textbf{x}_1) + \frac{1}{4}f(\textbf{x}_1)=$$

$$=f(\textbf{x}_1)$$

Иными словами, мы получили, что $f(\textbf{x}_1)\neq f(\textbf{x}_1)$, т.е. противоречие. На этом базу индукции можно считать доказанной.

II. Пусть обратное неравенство Йенсена верно для $n$ точек. Докажем для $n+1$. Далее предполагается, что $\lambda_{n+1}\neq 0$, иначе данное неравенство совпадает с неравенством для $n$ точек, которое верно по предположению индукции.

$$f\left(\lambda_1 \textbf{x}_1 +\dots+\lambda_n \textbf{x}_n + \lambda_{n+1}\textbf{x}_{n+1}\right) = $$
$$=f\left(\lambda_1 \textbf{x}_1 +\dots+(\lambda_n + \lambda_{n+1})\left(\frac{\lambda_n}{\lambda_n + \lambda_{n+1}} \textbf{x}_n + \frac{\lambda_{n+1}}{\lambda_n + \lambda_{n+1}}\textbf{x}_{n+1}\right)\right)=$$
$$=\sum\limits_{i=1}^{n-1}\lambda_i f(\textbf{x}_i) + \left(\lambda_n + \lambda_{n+1}\right)f\left(\frac{\lambda_n}{\lambda_n + \lambda_{n+1}} \textbf{x}_n + \frac{\lambda_{n+1}}{\lambda_n + \lambda_{n+1}}\textbf{x}_{n+1}\right) = $$
$$=\sum\limits_{i=1}^{n+1}\lambda_i f(\textbf{x}_i)$$

Стоит заметить, что $\lambda_n + \lambda_{n+1} < 0$ в силу условия и сделанного предположения отличности от нуля второго слагаемого. Из этого следует то, что $ \frac{\lambda_{n}}{\lambda_n + \lambda_{n+1}},  \frac{\lambda_{n+1}}{\lambda_n + \lambda_{n+1}} \geq 0$. Переход от предпоследней к последней строчки выполнен в силу выпуклости функции $f$. 

Полученное неравенство есть обратное неравенство Йенсена для $n+1$ точки.

Значит, обратное неравенство Йенсена верно. $\square$
\begin{center}
	\textbf{Задача 3.}
\end{center}

\begin{center}
	\textbf{3.1.}
\end{center}

Пользуясь ограничением на прямую, докажем, что $f$ - выпуклая функция.

Рассмотрим функцию для фиксированных $(t_1, \textbf{u}), (t_2, \textbf{v}) \in E$:

$$g(\alpha) = f((t_1, \textbf{u}) + \alpha (t_2, \textbf{v})),$$
$$\alpha \in\{\alpha|(t_1+\alpha t_2)^2 - (\textbf{u}+\alpha \textbf{v})^{\top}(\textbf{u}+\alpha \textbf{v})>0\}$$

Обозначим:

$$C_1 = t_2^2 - \textbf{v}^{\top}\textbf{v}$$
$$C_2 = t_1t_2 - \textbf{u}^{\top}\textbf{v}$$
$$C_3 = t_1^2 - \textbf{u}^{\top}\textbf{u}$$

В этих обозначениях получаем выражение для $g$:

$$g(\alpha) = -\log(C_1\alpha^2 + 2C_2\alpha + C_3)$$

Для определения выпуклости найдем ее вторую производную:

$$g''(\alpha) = \frac{2C_1}{C_1\alpha^2 + 2C_2\alpha + C_3}+4\frac{C_2^2-C_1C_3}{(C_1\alpha^2 + 2C_2\alpha + C_3)^2}$$

Из того, что $(t_2, \textbf{v}) \in E$, следует:

$$C_1 > 0, \forall (t_1, \textbf{u}), (t_2, \textbf{v}) \in E$$

Исходя из  определения области определения $g$, получаем неотрицательность знаменателей. Значит, если мы докажем, что числитель второго слагаемого в производной неотрицателен, то мы докажем, что вторая производная неотрицательна. Так займемся же этим.

$$C_2^2-C_1C_3=$$
$$= (t_1t_2 - (\textbf{u},\textbf{v}))^2 - (t_2^2 - \norm{\textbf{v}}^2)(t_1^2 - \norm{\textbf{u}}^2)=$$
$$=-2t_1t_2(\textbf{u},\textbf{v}) + (\textbf{u},\textbf{v})^2 + t_2^2\norm{\textbf{u}} + t_1^2\norm{\textbf{v}} -\norm{\textbf{v}}^2\norm{\textbf{u}}^2=$$
$$=(t_1\norm{\textbf{v}} - t_2\norm{\textbf{u}})^2+2t_1t_2(\norm{\textbf{v}}\norm{\textbf{u}}-(\textbf{u},\textbf{v})) + (\textbf{u},\textbf{v})^2 -\norm{\textbf{v}}^2\norm{\textbf{u}}^2=$$
$$=(t_1\norm{\textbf{v}} - t_2\norm{\textbf{u}})^2 + (\norm{\textbf{v}}\norm{\textbf{u}}-(\textbf{u},\textbf{v}))(2t_1t_2 - \norm{\textbf{v}}\norm{\textbf{u}}-(\textbf{u},\textbf{v}))\geq$$
$$\geq (\norm{\textbf{v}}\norm{\textbf{u}}-(\textbf{u},\textbf{v}))(2t_1t_2 - \norm{\textbf{v}}\norm{\textbf{u}}-(\textbf{u},\textbf{v}))$$
Пользуясь неравенством КБШ в обоих скобках, получаю:

$$C_2^2-C_1C_3 \geq 2c(t_1t_2 - \norm{\textbf{v}}\norm{\textbf{u}}),$$
где $c$ - неотрицательная константа. Скобка не меньше нуля, в силу определения множества $E$. Значит,

$$C_2^2-C_1C_3 \geq 0, \forall (t_1, \textbf{u}), (t_2, \textbf{v}) \in E$$

Как было отмечено выше, это означает неотрицательность второй производной $g$:

$$g''(\alpha) \geq 0, \forall (t_1, \textbf{u}), (t_2, \textbf{v}) \in E$$

Из чего следует, что $g$ выпукла для любых $(t_1, \textbf{u}), (t_2, \textbf{v}) \in E$. Тогда по теореме об ограничении на прямую исходная функция $f(t, \textbf{x})$ тоже выпукла.

\begin{center}
	\textbf{3.2.}
\end{center}

Область определения $p_{ij}$ есть множество векторов $X \in \mathbb{R}^n$ такое, что для любого вектора весов из этого множества выполняется, что любой цикл $G$, для которого есть его содержащий путь из $i$ в $j$, имеет неотрицательный вес. Если не выдвинуть этого условия, то вес кратчайшего пути может быть сколь угодно мал, а, значит,
 функция не определена. Отметим, что $X$ не пусто, т.к. $\mathbb{R}^n_+\in X$ (в таком случае функцию можно однозначно вычислить при помощи алгоритма Дейкстры).
 
 
 
Докажу, что функция $p_{ij}$ вогнута. 

Далее для вектора $\textbf{c}$ индекс внизу означает номер вектора, индекс вверху - номер компоненты этого вектора. Так же подразумевается, что $\alpha$ - произвольное число из отрезка $[0,1]$.

I.Рассмотрим два произвольных вектора $\textbf{c}_1,\textbf{c}_2 \in X$. Для этого сначала докажу, что $X$ - выпуклое множество. Заметим, что если $s_1, s_2$ -веса одного и того же пути $S$($S$ множество номеров ребер в этом пути) из одной вершины в другую(не обязательно $i$ и $j$) для векторов $\textbf{c}_1$ и $\textbf{c}_2$ соответственно, то вес этого пути для вектора $\alpha \textbf{c}_1 + (1-\alpha)\textbf{c}_2$ равен:

$$c(S) = \sum\limits_{i\in S}[\alpha \textbf{c}_1^i + (1-\alpha)\textbf{c}_2^i] = \alpha \sum\limits_{i\in S}\textbf{c}_1^i + (1-\alpha)\sum\limits_{i\in S}\textbf{c}_2^i = \alpha s_1 + (1-\alpha)s_2$$

Мы можем утверждать, что для любого цикла, для которого есть содержащий его путь из $i$ в $j$, верно, что его вес для вектора весов $\alpha \textbf{c}_1 + (1-\alpha)\textbf{c}_2$ неотрицателен. Действительно, это верно в силу написанного выше неравенства и того, что $\alpha \in[0,1]$ и для $с_1,c_2 \in X$ верно $s_1,s_2\geq 0$ по определению $X$, где $s_1,s_2$ - вес этого цикла для веторов $\textbf{c}_1,\textbf{c}_2$.

II. Введем обозначения: $S_1, S_2, S$ - множества номеров ребер в кратчайшем пути из $i$ в $j$ для векторов $\textbf{c}_1,\textbf{c}_2$ и $\alpha \textbf{c}_1 + (1-\alpha)\textbf{c}_2$ соответственно. В случае если кратчайших путей несколько, то выбирается любой.

В таких обозначениях, получаем следующие выражения для значений функций на этих векторах:

$$p_{ij}(\textbf{c}_k) = \sum\limits_{m \in S_k}c_k^m, k=1,2$$
$$p_{ij}(\alpha \textbf{c}_1 + (1-\alpha)\textbf{c}_2) = \sum\limits_{m \in S}\left(\alpha \textbf{c}_1^m + (1-\alpha)\textbf{c}_2^m\right) = \alpha\sum\limits_{m \in S}c_1^m+(1-\alpha)\sum\limits_{m \in S}c_2^m$$

По определению кратчайшего пути:

$$\sum\limits_{m \in S_1}c_1^m \leq \sum\limits_{m \in S}c_1^m$$

Из выше написанного следует:

$$p_{ij}(\alpha \textbf{c}_1 + (1-\alpha)\textbf{c}_2) \geq \alpha p_{ij}(\textbf{c}_1) + (1-\alpha)p_{ij}(\textbf{c}_2)$$

Что и означает то, что функция вогнута. $\square$
\end{document} 