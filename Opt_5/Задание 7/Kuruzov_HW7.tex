 \documentclass[12pt]{article}
\usepackage[T2A]{fontenc}
\usepackage[utf8]{inputenc}       

\usepackage[english,russian]{babel}
\usepackage{amsmath,amsfonts,amsthm,amssymb,amsbsy,amstext,amscd,amsxtra,multicol}
\usepackage{verbatim}
\usepackage{tikz}
\usetikzlibrary{automata,positioning}
\usepackage{multicol}
\usepackage{graphicx}
\usepackage[colorlinks,urlcolor=blue]{hyperref}
\usepackage[stable]{footmisc}
\usepackage{ dsfont }

\usepackage{xparse}
\usepackage{ifthen}
\usepackage{bm}
\usepackage{color}

\usepackage{algorithm}
\usepackage{algpseudocode}

\DeclareMathOperator*{\argmin}{argmin}
\DeclareMathOperator*{\argmax}{argmax}
\DeclareMathOperator*{\sign}{sign}
\DeclareMathOperator*{\aff}{aff}
\DeclareMathOperator*{\conv}{conv}
\DeclareMathOperator*{\relint}{relint}
\DeclareMathOperator*{\intset}{int}
\DeclareMathOperator*{\dom}{dom}
\DeclareMathOperator*{\tr}{tr}

\newcommand\norm[1]{\left\lVert#1\right\rVert}
\newcommand\abs[1]{\left|#1\right|}
\begin{document}

\begin{center}
	{Курузов Илья, 678}

	{Задание 7}
\end{center}

\begin{center}
	\textbf{Задача 1.}
\end{center}

1) Пусть $p>1$. Тогда найдем супремум при помощи производной.

$$(yx - x^p)'=0$$
$$y - px^{p-1}=0$$
$$x* = \left(\frac{y}{p}\right)^{\frac{1}{p-1}}$$

Данная точка дает максимум:

$$f^*(y)=y\left(\frac{y}{p}\right)^{\frac{1}{p-1}} + \left(\frac{y}{p}\right)^{\frac{p}{p-1}}, \dom f^* = \mathbb{R}$$

2) Пусть $p<0$. Тогда рассмотрим 2 случая:

a) $y\neq 0$. Тогда $\sup (yx - x^p) = (yx - x^p)|_{x = \sign(y) \infty} = \infty$. Значит, при $y \neq 0$ сопряженная функция не определена.

б) $y = 0$. Тогда $\sup(yx-x^p) = -\inf(x^p) = 0$

Тогда 
$$\boxed{f^*(y) = 0, \dom f^* = \{0\}}$$

\begin{center}
	\textbf{Задача 2.}
\end{center}

$$f^*(\textbf{y}) = \sup\limits_{x_j> 0,\forall j}\left(\sum\limits_{j=1}^n y_jx_j + \left(\prod\limits_{j=1}^nx_j\right)^{\frac{1}{n}}\right)$$

1) Если $\exists j:y_j\geq 0$, то 
$$\sup\limits_{x_j> 0,\forall j}\left(\sum\limits_{j=1}^n y_jx_j + \left(\prod\limits_{j=1}^nx_j\right)^{\frac{1}{n}}\right) = \lim\limits_{x_j\to \infty}\left(\sum\limits_{j=1}^n y_jx_j + \left(\prod\limits_{j=1}^nx_j\right)^{\frac{1}{n}}\right)\Big|_{x_i = 1,\forall i \neq j} = \infty$$

Значит, если $\exists j:y_j\geq 0$, то функция не определена.

2) Рассмотрим случай, когда $\sum\limits_{j=1}^n y_j > -1$. Докажу, что в этом случае сопряженная функция не определена. Положим, все компоненты вектора $\textbf{x}$ равными $m>0$ и рассмотрим выражение под супремумом

\begin{eqnarray}
\sum\limits_{j=1}^n y_jx_j + \left(\prod\limits_{j=1}^nx_j\right)^{\frac{1}{n}} = \\
= m\left(\sum\limits_{j=1}^n y_j + 1\right) = \\
= mq,
\end{eqnarray}

где $q > 0$. В силу того, что $m$ - произвольная положительная константа, верно, что выражение под супремумом может принимать сколь угодно большое значение, а, значит, в этом случае супремум бесконечен и функция не определена.

3) Теперь рассмотрим случай, когда $y_i \leq -\frac{1}{n}, \forall i$.

\begin{eqnarray}
\sum\limits_{j=1}^n y_jx_j + \left(\prod\limits_{j=1}^nx_j\right)^{\frac{1}{n}} \leq \\
\leq \sum\limits_{j=1}^n y_jx_j + \sum\limits_{j = 1}^n \frac{x_j}{n} = \\
= \sum\limits_{j=1}^n \left(y_j+\frac{1}{n}\right)x_j \leq \\
\leq \min\limits_j x_j \left(\sum\limits_{j=1}^n y_j + 1\right) \leq 0
\end{eqnarray}

Переход от первой ко второй строчке осуществлен в силу неравенства о средних, а неравенство в четвертой строчке осуществлено в силу области определения $y_i$. Верхняя оценка достигается при $\textbf{x}\to\textbf{0}$. Значит, 

$$f(\textbf{y})\Big|_{\textbf{y}|y_i \leq -\frac{1}{n}, \forall i}= 0, \dom f^* = \left\{\textbf{y}|y_i \leq -\frac{1}{n}, \forall i\right\}$$

\begin{center}
	\textbf{Задача 3.}
\end{center}

По определению:

$$g^*(\textbf{y}) = \sup\limits_{\textbf{x}}\left(\textbf{y}^\top \textbf{x} - g(\textbf{x})\right)$$

$$g^*(\textbf{y}) = \sup\limits_{\textbf{x}}\left(\textbf{y}^\top \textbf{x} - f(\textbf{Ax}+\textbf{b})\right)$$

Пусть далее $A$ не вырожденная матрица.

\begin{eqnarray}
\textbf{y}^\top \textbf{x} =\nonumber \\
= \textbf{y}^\top \textbf{A}^{-1} \left(\textbf{A}\textbf{x}\right)=\nonumber\\
= \textbf{y}^\top \textbf{A}^{-1} \left(\textbf{A}\textbf{x} + \textbf{b} - \textbf{b}\right)=\nonumber\\
= (\textbf{A}^{-\top}\textbf{y})^\top\left(\textbf{A}\textbf{x} + \textbf{b}\right)  - \textbf{y}^\top \textbf{A}^{-1} \textbf{b}
\end{eqnarray}

В результате получаем:

$$g^*(\textbf{y}) = \sup\limits_{\textbf{x}}\left((\textbf{A}^{-\top}\textbf{y})^\top\left(\textbf{A}\textbf{x} + \textbf{b}\right) - f(\textbf{Ax}+\textbf{b})\right) - \textbf{y}^\top \textbf{A}^{-1} \textbf{b}$$

$$g^*(\textbf{y}) = \sup\limits_{\textbf{x}}\left((\textbf{A}^{-\top}\textbf{y})^\top\textbf{x} - f(\textbf{x})\right) - \textbf{y}^\top \textbf{A}^{-1} \textbf{b}$$

$$\boxed{g^*(\textbf{x}) = f^*(\textbf{A}^{-\top}\textbf{y}) - \textbf{y}^\top \textbf{A}^{-1} \textbf{b}}$$

\begin{center}
	\textbf{Задача 4.}
\end{center}

$$f^*(\textbf{Y}) = \sup\limits_{\textbf{X} \in \textbf{S}^n_{++}}\left(\tr(\textbf{YX}) + \log\det\textbf{X} \right)$$

Найдем производную выражения под супремумом и приравняем ее нулю. В силу того, что производные слагаемых уже были найдены в курсе, воспользуемс выражениями для них без приведения вывода:

$$\left(\tr(\textbf{YX}) + \log\det\textbf{X} \right)' = \textbf{Y}^{\top} + X^{-\top} = 0$$

1) Рассмотрим случай, когда $Y$ - невырожденная матрица. Тогда $X = -Y^{-1}$ дает экстремум, причем, в силу выпуклости функций (доказано в курсе), - глобальный. Данный экстремум дает максимум.

$$f^*(\textbf{Y}) = -n + \log\det(-\textbf{Y}^{-1})$$

$$f^*(\textbf{Y}) = -n - \log\det(-\textbf{Y}), \forall \textbf{Y}: \det(-\textbf{Y}) > 0$$

2) Пусть теперь $Y$ - матрица, такая что $\det(-\textbf{Y})\leq 0$. Докажу, что в этом случае сопряженная функция не определена. Для случая если на диагонали $\textbf{Y}$ есть неотрицательный элемент (без ограничения общности будем считать, что это $y_{11}$), то рассмотрим выражение под супремумом для $\textbf{X}=diag(k\lambda_1\dots \lambda_n)$, где $k>0$, $\lambda_i>0, \forall i$. Заметим, что $\textbf{X} \in \textbf{S}^n_{++}$.

$$\tr(\textbf{YX}) + \log\det\textbf{X} = k\lambda_1 y_1 +\sum\limits_{j=2}^ny_i \lambda_i + \log \prod_{j=1}^n\lambda_j+\log k$$

Как видим, устремляя $k$ к бесконечности, получаем сколь угодно большое значение для выражения под супремумом. А, значит, супремум бесконечен и функция не определена.

\end{document}