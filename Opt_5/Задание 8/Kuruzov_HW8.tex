 \documentclass[12pt]{article}
\usepackage[T2A]{fontenc}
\usepackage[utf8]{inputenc}       

\usepackage[english,russian]{babel}
\usepackage{amsmath,amsfonts,amsthm,amssymb,amsbsy,amstext,amscd,amsxtra,multicol}
\usepackage{verbatim}
\usepackage{tikz}
\usetikzlibrary{automata,positioning}
\usepackage{multicol}
\usepackage{graphicx}
\usepackage[colorlinks,urlcolor=blue]{hyperref}
\usepackage[stable]{footmisc}
\usepackage{ dsfont }

\usepackage{xparse}
\usepackage{ifthen}
\usepackage{bm}
\usepackage{color}

\usepackage{algorithm}
\usepackage{algpseudocode}

\DeclareMathOperator*{\argmin}{argmin}
\DeclareMathOperator*{\argmax}{argmax}
\DeclareMathOperator*{\sign}{sign}
\DeclareMathOperator*{\aff}{aff}
\DeclareMathOperator*{\conv}{conv}
\DeclareMathOperator*{\relint}{relint}
\DeclareMathOperator*{\intset}{int}
\DeclareMathOperator*{\dom}{dom}
\DeclareMathOperator*{\tr}{tr}

\newcommand\norm[1]{\left\lVert#1\right\rVert}
\newcommand\abs[1]{\left|#1\right|}
\begin{document}

\begin{center}
	{Курузов Илья, 678}

	{Задание 8}
\end{center}

\begin{center}
	\textbf{Задача 1}
\end{center}

Найдем условный субдифференциал:

\begin{equation}
\begin{split}
\partial_X f(x,y) = &\partial f(x,y) + N((x,y)|X) = \\
&=\begin{cases}
(1, 0)^\top, \text{if $x > 0$}\\
(\alpha, 0)^\top, \alpha \in[-1,1], \text{if $x = 0$}
\end{cases} +\\
&+
\begin{cases}
(0, 1)^\top, \text{if $y > 0$}\\
(0, \alpha)^\top, \alpha \in[-1,1], \text{if $y = 0$}
\end{cases}
+N((x,y)|X)=\\
&=\begin{cases}
(1, 0)^\top, \text{if $x > 0$}\\
(\alpha, 0)^\top, \alpha \in[-1,1], \text{if $x = 0$}
\end{cases} +\\
&+
\begin{cases}
(0, 1)^\top, \text{if $y > 0$}\\
(0, \alpha)^\top, \alpha \in[-1,1], \text{if $y = 0$}
\end{cases}+\\
&+\begin{cases}
\alpha \textbf{n}, \alpha > 0, \text{if $(x-1)^2+(y-1)^2 =1$}\\
0, \text{else}
\end{cases},
\end{split}
\end{equation}
где $\textbf{n} = (x-1, y-1)^\top$ - внешняя нормаль к окружности  в точке $(x,y)$. Выше не расматривались случаи $x<0$ и $y<0$, т.к. такие точки не удовлетворяют ограничению.

1) На границе.
1.1)$x > 0, y > 0$

\begin{equation}
\begin{cases}
1 + \alpha(x-1)= 0,\\
1 + \alpha(y-1)= 0
\end{cases}
\end{equation}

\begin{equation}
\begin{cases}
x= 1 - \frac{1}{\alpha},\\
y = 1 - \frac{1}{\alpha}
\end{cases}
\end{equation}
 
Найдем $\alpha$ из того условия, что это граница:

$$\frac{2}{\alpha^2}=1$$
$$\alpha = \sqrt{2}$$ 

$$(x^*, y^*)^\top = (1 - \frac{1}{\sqrt{2}}, 1 - \frac{1}{\sqrt{2}})^\top$$

В силу выпуклости целевой функции и бюджетного множества, можно утверждать, что в точке $(x^*, y^*)^\top = (1 - \frac{1}{\sqrt{2}}, 1 - \frac{1}{\sqrt{2}})^\top$ достигается минимум, равный $2 -\sqrt{2}$.

\begin{center}
	\textbf{Задача 2.}
\end{center}

1) Исследуем на выпуклость целевую функцию.

$$H(f) = \begin{Vmatrix}
2 + e^{x_1+x_2} & -1 + e^{x_1+x_2}\\
-1 + e^{x_1+x_2} & 4 + e^{x_1+x_2}
\end{Vmatrix}
$$

$$\Delta_1 = 2 + e^{x_1+x_2} > 0, \forall \textbf{x}\in\mathbb{R}^2$$
$$\Delta_2 = (2 + e^{x_1+x_2})(4 + e^{x_1+x_2}) - (-1 + e^{x_1+x_2})^2 = 7 +7e^{x_1+x_2} > 0, \forall \textbf{x}\in\mathbb{R}^2$$

Значит, целевая функция действительно выпуклая. Значит, любая стационарная точка даст нам минимум.

2) $$\textbf{grad}f = \textbf{0}$$
\begin{equation}
\begin{cases}
2x_1 - x_2 + e^{x_1+x_2} = 0,\\
-x_1 + 4x_2 + e^{x_1+x_2} = 0.
\end{cases}
\end{equation}

Решение этой системы - $(-0.385, -0.231)$ дает минимум, равный $0.706$.

\begin{center}
	\textbf{Задача 3}
\end{center}

Построим лагранжиан:

$$L(x,y,\lambda,\mu) = (x-3)^2 - (y-2)^2 + \lambda(y-x-1)+\mu(y+x-3)$$

Используем теорему Каруша-Куна-Такера:
\begin{equation}
\begin{cases}
y-x-1 = 0,\\
y+x-3 \leq 0,\\
\mu(y+x-3) = 0,\\
\mu \geq 0,\\
2(x-3)-\lambda+\mu=0,\\ 
-2(y-2) +\lambda+\mu = 0.
\end{cases}
\end{equation}

1) $\mu = 0$

\begin{equation}
\begin{cases}
y-x-1 = 0,\\
y+x-3 \leq 0,\\
2(x-3)-\lambda=0,\\ 
-2(y-2) +\lambda = 0;
\end{cases}
\end{equation}

\begin{equation}
\begin{cases}
y-x-1 = 0,\\
x - y - 1=0,\\ 
x + y - 3 \leq 0,\\
-2(y-2) +\lambda = 0;
\end{cases}
\end{equation}

Система не совместна.

2) $\mu > 0$

\begin{equation}
\begin{cases}
y-x-1 = 0,\\
y+x-3 = 0,\\
\mu \geq 0,\\
2(x-3)-\lambda+\mu=0,\\ 
-2(y-2) +\lambda+\mu = 0;
\end{cases}
\end{equation}

\begin{equation}
\begin{cases}
y-x-1 = 0,\\
x-y-1+\mu = 0
y+x-3 = 0,\\
\mu \geq 0,\\
2(x-3)-\lambda+\mu=0,\\ 
\end{cases}
\end{equation}

\begin{equation}
\begin{cases}
\mu = 2, \\
y = 2,\\
x = 1
\mu \geq 0,\\
\end{cases}
\end{equation}

Точка $(1,2)$ - стационарная точка, а в силу выпуклости задачи, эта точка дает минимум.

$$\boxed{\min\limits_X f(x,y) = f(1, 2)= 4}$$

\begin{center}
	\textbf{Задача 4}
\end{center}

Перепишем оптимизационную задачу в следующем виде:

\begin{gather}
\min\limits_{\textbf{x}\in\mathbb{R}^3}x_1^2+2x_2^2 +x_3\\
s.t. x_1 - 2x_2+3x_3 -4\leq 0\\
-(x_1 - 2x_2+3x_3) -4\leq 0
\end{gather}

Целевая функция и функции, задающие ограничения, дифференцируемы. Применим теорему Каруша-Куна-Такера:

$$L = x_1^2+2x_2^2 +x_3 + \mu_1(x_1 - 2x_2+3x_3 -4) + \mu_2(-(x_1 - 2x_2+3x_3) -4)$$

\begin{equation}
\begin{cases}
x_1 - 2x_2+3x_3 -4\leq 0\\
-(x_1 - 2x_2+3x_3) -4\leq 0,\\
 \mu_1(x_1 - 2x_2+3x_3 -4) = 0,\\
\mu_2(-(x_1 - 2x_2+3x_3) -4) = 0,\\
\mu_1,\mu_2\geq 0,\\
2x_1 + \mu_1-\mu_2=0,\\
4x_2- 2\mu_1+2\mu_2=0,\\
1 + 3\mu_1 - 3\mu_2 = 0.
\end{cases}
\end{equation}

1) $\mu_1 = 0, \mu_2 > 0$.

\begin{equation}
\begin{cases}
x_1 - 2x_2+3x_3 -4\leq 0\\
-(x_1 - 2x_2+3x_3) -4= 0,\\
\mu_1,\mu_2\geq 0,\\
2x_1-\mu_2=0,\\
4x_2+2\mu_2=0,\\
1 - 3\mu_2 = 0.
\end{cases}
\end{equation}

\begin{equation}
\begin{cases}
x_1 - 2x_2+3x_3 -4\leq 0\\
x_3= -\frac{3}{2},\\
\mu_2\geq 0,\\
x_1=\frac{1}{6},\\
x_2=-\frac{1}{6},\\
\mu_2 = \frac{1}{3}.
\end{cases}
\end{equation}

Точка $(\frac{1}{6}, -\frac{1}{6}, -\frac{3}{2})^\top$ - решение этой системы. В силу выпуклости задачи, она дает минимум.

$$\boxed{\min\limits_{\textbf{x}\in X }x_1^2+2x_2^2 +x_3 = x_1^2+2x_2^2 +x_3\Big|_{\textbf{x}=(\frac{1}{6}, -\frac{1}{6}, -\frac{3}{2})^\top} = -\frac{13}{9}}$$

\end{document}