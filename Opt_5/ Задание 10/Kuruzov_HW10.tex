 \documentclass[12pt]{article}
\usepackage[T2A]{fontenc}
\usepackage[utf8]{inputenc}       

\usepackage[english,russian]{babel}
\usepackage{amsmath,amsfonts,amsthm,amssymb,amsbsy,amstext,amscd,amsxtra,multicol}
\usepackage{verbatim}
\usepackage{tikz}
\usepackage{pgfplots}
\pgfplotsset{compat=1.9}
\usetikzlibrary{automata,positioning}
\usepackage{multicol}
\usepackage{graphicx}
\usepackage[colorlinks,urlcolor=blue]{hyperref}
\usepackage[stable]{footmisc}
\usepackage{ dsfont }

\usepackage{xparse}
\usepackage{ifthen}
\usepackage{bm}
\usepackage{color}

\usepackage{algorithm}
\usepackage{algpseudocode}

\DeclareMathOperator*{\argmin}{argmin}
\DeclareMathOperator*{\argmax}{argmax}
\DeclareMathOperator*{\sign}{sign}
\DeclareMathOperator*{\aff}{aff}
\DeclareMathOperator*{\conv}{conv}
\DeclareMathOperator*{\relint}{relint}
\DeclareMathOperator*{\intset}{int}
\DeclareMathOperator*{\dom}{dom}
\DeclareMathOperator*{\tr}{tr}
\DeclareMathOperator*{\st}{s.t. }
\newcommand\norm[1]{\left\lVert#1\right\rVert}
\newcommand\abs[1]{\left|#1\right|}
\begin{document}

\begin{center}
	{Курузов Илья, 678}

	{Задание 10}
\end{center}

\begin{center}
	\textbf{Задача 1}
\end{center}

Заметим, что задача $\min |x|$ равносильна следующей задаче:

\begin{gather}
\min y\\
\st\,y\geq x\\
y\geq -x
\end{gather} 

Заметим, что добавление условий на неотрицательность переменных не изменит решения этой задачи. Далее это свойство будет активно использоваться

а) 

\begin{gather}
\min 2x_1+3|x_2-10|\\
\st\,|x_1+2|+|x_2|\leq 5
\end{gather} 

Заменим $|x_2-10|,|x_1+2|,|x_2|$ на $y_1, y_2,y_3$.

\begin{gather}
\min 2x_1+3y_1\\
\st\,y_2+y_3\leq 5\\
y_1 \geq x_2 - 10\\
y_1 \geq -(x_2 - 10)\\
y_2 \geq x_1 + 2\\
y_2 \geq -(x_1+2)\\
y_3 \geq x_2\\
y_3 \geq -x_2\\
y_1, y_2, y_3 \geq 0\\
x_1+2, x_2, x_2-10 \geq 0
\end{gather} 

Или полностью по канону:

\begin{gather}
\min 2x_1+3y_1\\
\st\,y_2+y_3\leq 5\\
y_1 \geq x_2 - 10\\
y_1 \geq -(x_2 - 10)\\
y_2 \geq x_1 + 2\\
y_2 \geq -(x_1+2)\\
y_3 \geq x_2\\
y_3 \geq -x_2\\
x_2 \geq 10\\
x_1 - v \geq -2\\
v \leq 2\\
y_1, y_2, y_3 \geq 0\\
x_1, x_2, v\geq 0
\end{gather} 

Переходим к равенствам и записываем полностью в каноническом виде

\begin{gather}
\min 2x_1+3y_1\\
\st\,y_2+y_3\leq 5\\
y_1 - x_2 + z_1 = - 10\\
y_1 + x_2 + z_2 = 10)\\
y_2 - x_1 + z_3 = 2\\
y_2 + x_1 + z_4 =  -2\\
y_3 - x_2 + z_5 = 0\\
y_3 + x_2 + z_6 = 0\\
x_2 + z_7 = 10\\
x_1 - v  + z_8 = -2\\
-v + z_9 = -2\\
y_1, y_2, y_3 \geq 0\\
x_1, x_2, v\geq 0\\
\textbf{z} \geq 0
\end{gather} 

б) 

\begin{gather}
\min \norm{\textbf{x}}\\
\st\,\textbf{Ax} = \textbf{b}
\end{gather} 

Переходим от $|x_i|$ к $y_i$.

\begin{gather}
\min \sum\limits_{i=1}^ny_i\\
\st\,\textbf{Ax} = \textbf{b}\\
y_i \geq x_i\\
y_i \geq -x_i
\end{gather}

Замена $\textbf{x} = \textbf{u} - \textbf{v}, \textbf{u}\geq 0, \textbf{v} \geq 0$

\begin{gather}
\min \sum\limits_{i=1}^ny_i\\
\st\,\textbf{Au}-\textbf{Av} = \textbf{b}\\
\textbf{y} \geq \textbf{u} - \textbf{v}\\
\textbf{y} \geq -\textbf{u} + \textbf{v}\\
\textbf{u}, \textbf{v}, \textbf{y} \geq 0
\end{gather}

Перейдем к канонической форме:

\begin{gather}
\min \sum\limits_{i=1}^ny_i\\
\st\,\textbf{Au}-\textbf{Av} = \textbf{b}\\
\textbf{y} - \textbf{u} + \textbf{v} + \textbf{z}_1 = 0\\
\textbf{y} + \textbf{u} - \textbf{v} + \textbf{z}_2 =0\\
\textbf{u}, \textbf{v}, \textbf{y}, \textbf{z}_1, \textbf{z}_2 \geq 0
\end{gather}

в) Полиэдральное множество $P = \left\{\textbf{x}\Big|\textbf{Ax}\leq \textbf{b}\right\}$

\begin{gather}
\max\limits_{\textbf{x}_c}R\\
\st\,B(\textbf{x}_c, R)\subset P
\end{gather}

Шар $B(\textbf{x}_c, R)$ лежит в $P$, тогда и только тогда, когда граница шара лежит в $P$. Необходимость следует из определения вложенности в одного множества в другое. Достаточность можно доказать от противного.

Радиус наибольшей сферы с центром $\textbf{x}_c$ равен минимальному расстоянию от $\textbf{x}_c$ до $P$. Теперь получаем следующую эквивалентную оптимизационную задачу:

\begin{gather}
\min\limits_{\textbf{x}_c}\norm{\textbf{x} - \textbf{x}_c}\\
\st\, \textbf{x}\in P
\end{gather}

Выше и далее под нормой подразумевается вторая норма. Очевидное замечание: $x_c$ должно лежать в $P$.

\begin{gather}
\min\norm{\textbf{x} - \textbf{x}_c}\\
\st\, \textbf{x}, \textbf{x}_c\in P
\end{gather}

\begin{gather}
\min\norm{\textbf{x} - \textbf{x}_c}\\
\st\, \textbf{Ax}, \textbf{Ax}_c \leq \textbf{b}
\end{gather}
\end{document}