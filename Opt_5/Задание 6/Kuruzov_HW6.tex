 \documentclass[12pt]{article}
\usepackage[T2A]{fontenc}
\usepackage[utf8]{inputenc}       

\usepackage[english,russian]{babel}
\usepackage{amsmath,amsfonts,amsthm,amssymb,amsbsy,amstext,amscd,amsxtra,multicol}
\usepackage{verbatim}
\usepackage{tikz}
\usetikzlibrary{automata,positioning}
\usepackage{multicol}
\usepackage{graphicx}
\usepackage[colorlinks,urlcolor=blue]{hyperref}
\usepackage[stable]{footmisc}
\usepackage{ dsfont }

\usepackage{xparse}
\usepackage{ifthen}
\usepackage{bm}
\usepackage{color}

\usepackage{algorithm}
\usepackage{algpseudocode}

\DeclareMathOperator*{\argmin}{argmin}
\DeclareMathOperator*{\argmax}{argmax}
\DeclareMathOperator*{\sign}{sign}
\DeclareMathOperator*{\aff}{aff}
\DeclareMathOperator*{\conv}{conv}
\DeclareMathOperator*{\relint}{relint}
\DeclareMathOperator*{\intset}{int}
\DeclareMathOperator*{\dom}{dom}

\newcommand\norm[1]{\left\lVert#1\right\rVert}
\newcommand\abs[1]{\left|#1\right|}
\begin{document}

\begin{center}
	{Курузов Илья, 678}

	{Задание 6}
\end{center}

\begin{center}
	\textbf{Задача 1.}
\end{center}

$$\norm{\textbf{x}} = \sum\limits_{j=1}^n g(x_j)$$

$$g(\textbf{x}) = |x_j|$$

\begin{equation}
\partial g(\textbf{x}) = 
\begin{cases}
-e_j, \text{if $x < 0$} \\
[-1, 1]e_j, \text{if $x = 0$} \\
e_j, \text{if $x > 0$} \\
\end{cases},
\end{equation}
где $e_j$ - $j$-ый базисный вектор.

\begin{equation}
\partial\norm{\textbf{x}} = \sum\limits_{j=1}^n \partial g(\textbf{x}) =
\sum\limits_{j=1}^n
\begin{cases}
-e_j, \text{if $x_j < 0$} \\
[-1, 1]e_j, \text{if $x_j = 0$} \\
e_j, \text{if $x_j > 0$} \\
\end{cases} 
\end{equation}

\begin{center}
	\textbf{Задача 2.}
\end{center}


\begin{equation}
f(x) = \frac{1}{2}\norm{\textbf{w}}_2^2 + \sum\limits_{i=1}^m \max \left(0, 1-y_i((\textbf{w},\textbf{x}_i) + b_i)\right)
\end{equation}

Найдем субградиент для выражения под суммой:

\begin{eqnarray}
\partial \max \left(0, 1-y_i((\textbf{w},\textbf{x}_i) + b_i)\right) =\nonumber\\
=\begin{cases}
\partial 0, \text{if $0 > 1-y_i((\textbf{w},\textbf{x}_i) + b_i)$} \\
\partial \left(1-y_i((\textbf{w},\textbf{x}_i) + b_i)\right), \text{if $0 < 1-y_i((\textbf{w},\textbf{x}_i) + b_i)$} \\
\conv\left(0, \partial(1-y_i((\textbf{w},\textbf{x}_i) + b_i))\right), \text{if $0 = 1-y_i((\textbf{w},\textbf{x}_i) + b_i)$} \\
\end{cases}
\end{eqnarray}


\begin{eqnarray}
\partial \max \left(0, 1-y_i((\textbf{w},\textbf{x}_i) + b_i)\right) = \nonumber\\
=\begin{cases}
0, \text{if $0 > 1-y_i((\textbf{w},\textbf{x}_i) + b_i)$} \\
-y_i\textbf{x}_i, \text{if $0 < 1-y_i((\textbf{w},\textbf{x}_i) + b_i)$} \\
\conv\left(0, -y_i\textbf{x}_i)\right), \text{if $0 = 1-y_i((\textbf{w},\textbf{x}_i) + b_i)$} \\
\end{cases}
\end{eqnarray}

\begin{equation}
\frac{1}{2}\partial\norm{\textbf{w}}_2^2 = \textbf{w}
\end{equation}

Конечное выражение для субдифференциала

\begin{equation}
\boxed{\partial f(x) = \textbf{w} +
\sum\limits_{j=1}^m
\begin{cases}
\textbf{0}, \text{if $0 > 1-y_i((\textbf{w},\textbf{x}_i) + b_i)$} \\
-y_i\textbf{x}_i, \text{if $0 < 1-y_i((\textbf{w},\textbf{x}_i) + b_i)$} \\
-y_i\alpha\textbf{x}_i, \alpha\in[0, 1], \text{if $0 = 1-y_i((\textbf{w},\textbf{x}_i) + b_i)$} \\
\end{cases}}
\end{equation}

\begin{center}
	\textbf{Задача 3.}
\end{center}

\begin{equation}
f(x)=
=\begin{cases}
0, \text{if $x \in [-1, 1]$}\\
|x|-1, \text{if $x \in [-2, -1) \cup (1, 2]$}\\
-\infty, \text{else}\\
\end{cases}
\end{equation}

Найдем субдифференциал функции на интервале (-2, 2):

\begin{equation}
\partial_{(-2, 2)} f(x)=
\begin{cases}
1, \text{if $x \in (1, 2)$}\\
\conv\left(0, 1\right), \text{if $x = 1$}\\
0, \text{if $x \in (-1, 1)$}\\
\conv\left(0, -1\right), \text{if $x = -1$}\\
-1, \text{if $x \in (-2, -1)$}\\
\end{cases}
\end{equation}

Пусть $X = [-2, 2]$, тогда искомый субдифференциал на $X$:

\begin{equation}
\partial_X f(x) = \partial_{(-2, 2)}f(x) + N(x|X)
\end{equation}

\begin{equation}
\partial N(x|X) =
\begin{cases}
a, a\geq 0, \text{if $x = -2$}\\
0, \text{if $x \in (-\infty, \infty)\backslash \{-2, 2\}$}\\
a, a\leq 0, \text{if $x = 2$}\\
\end{cases}
\end{equation}


\begin{equation}
\boxed{
\partial_X f(x) = 
\begin{cases}
a, a\leq 0, \text{if $x = 2$}
1, \text{if $x \in (1, 2)$}\\
\alpha, \alpha \in[0, 1], \text{if $x = 1$}\\
0, \text{if $x \in (-\infty, -2)\cap(-1, 1)\cap(2,\infty)$}\\
\alpha, \alpha \in[-1, 0], \text{if $x = -1$}\\
-1, \text{if $x \in (-2, -1)$}\\
a, a\geq 0, \text{if $x = -2$}
\end{cases}}
\end{equation}


\begin{center}
	\textbf{Задача 4.}
\end{center}

\begin{equation}
\partial_X f(\textbf{x}) = \partial f(\textbf{x}) + \partial \delta(x|X)= \partial f(\textbf{x}) + N(x|X) 
\end{equation}

Найдем $\partial f(\textbf{x})$:

\begin{eqnarray}
\partial f(\textbf{x}) = \partial |x_1-x_2| + \partial |x_1+x_2| = \nonumber\\
= \begin{cases}
(1, -1)^{\top}, \text{$x_1>x_2$}\\
\conv\{(1, -1)^{\top}, (-1, 1)^{\top}\}, \text{$x_1=x_2$}\\
(-1, 1)^{\top}, \text{$x_1<x_2$}\\
\end{cases} + 
\begin{cases}
(1, 1)^{\top}, \text{$x_1 +x_2>0$}\\
\conv\{(1, 1)^{\top}, (-1, -1)^{\top}\}, \text{$x_1=x_2$}\\
(-1, -1)^{\top}, \text{$x_1+x_2<0$}\\
\end{cases}
\end{eqnarray}

Нормальный конус:

\begin{equation}
N(x|X) = 
\begin{cases}
\{a\textbf{x}| a> 0\}, \text{$\norm{\textbf{x}} = 2$}\\
\textbf{0}, else\\
\end{cases}
\end{equation}

Окончательный ответ:

\begin{eqnarray}
\partial_X f(\textbf{x}) = 
\begin{cases}
(1, -1)^{\top}, \text{$x_1>x_2$}\\
\conv\{(1, -1)^{\top}, (-1, 1)^{\top}\}, \text{$x_1=x_2$}\\
(-1, 1)^{\top}, \text{$x_1<x_2$}\\
\end{cases} + \nonumber\\
\begin{cases}
(1, 1)^{\top}, \text{$x_1 +x_2>0$}\\
\conv\{(1, 1)^{\top}, (-1, -1)^{\top}\}, \text{$x_1=-x_2$}\\
(-1, -1)^{\top}, \text{$x_1+x_2<0$}\\
\end{cases} +\nonumber\\
\begin{cases}
\{a\textbf{x}| a> 0\}, \text{$\norm{\textbf{x}} = 2$}\\
\textbf{0}, else\\
\end{cases}
\end{eqnarray}



\end{document} 