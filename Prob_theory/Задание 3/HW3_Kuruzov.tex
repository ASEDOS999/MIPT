 \documentclass[12pt]{article}
\usepackage[T2A]{fontenc}
\usepackage[utf8]{inputenc}       

\usepackage[english,russian]{babel}
\usepackage{amsmath,amsfonts,amsthm,amssymb,amsbsy,amstext,amscd,amsxtra,multicol}
\usepackage{verbatim}
\usepackage{tikz}
\usetikzlibrary{automata,positioning}
\usepackage{multicol}
\usepackage{graphicx}
\usepackage[colorlinks,urlcolor=blue]{hyperref}
\usepackage[stable]{footmisc}
\usepackage{ dsfont }

\usepackage{xparse}
\usepackage{ifthen}
\usepackage{bm}
\usepackage{color}

\usepackage{algorithm}
\usepackage{algpseudocode}

\DeclareMathOperator*{\argmin}{argmin}
\DeclareMathOperator*{\argmax}{argmax}
\DeclareMathOperator*{\sign}{sign}
\newcommand\norm[1]{\left\lVert#1\right\rVert}

\begin{document}

\begin{center}
{Курузов Илья}

{Задание 3}
\end{center}

\begin{center}
\textbf{Задача 1.}
\end{center}

1)

$$F_{\max}(x) = \mathbb{P}(\max\{X,Y\}<x)$$
$$F_{\min}(x) = \mathbb{P}(\min\{X,Y\}<x)$$

Каждое из условий $\max\{X,Y\}<x$ и $\max\{X,Y\}<x$ равносильно условию $X<x, Y<x$.

Из независимости случайных величин:
$$F_{\max}(x)=F_{\min}(x) = \mathbb{P}(X<x, Y<y) = F_X(x)F_Y(y)$$

2) $$F_Y(x) = \mathbb{P}(Y < x)=\mathbb{P}(F_X(X(\omega)) < x)$$

Для написания аналитического выражения для распределения $Y$ требуется уточнение конструкции множества элементарных исходов.

3) Исходя из монотонности $F$:

$$F^{-1}(y) < z \Leftrightarrow F(F^{-1}(y)) < F(z)$$

Далее найдем распределение:

$$F_{new}(x) = \mathbb{P}(F^{-1}(\xi)<x)=\mathbb{P}(F(F^{-1}(\xi))<F(x))$$

Для всеx $\xi$, для которых $F$ в точке $F^{-1}(\xi)$ непрерывно, выполнено, что $F(F^{-1}(\xi)) = \xi$. Множество точек разрывов имеет меру нуль(следует из монотонности функции). Значит, если переопределить в этих точках сложную фунцию $F(F^{-1}(\xi)):=\xi$, то значение вероятности не изменится.

$$F_{new}(x) = \mathbb{P}(\xi<F(x)) = F_{\xi}(F(x))=F(x)$$

Что и требовалось доказать.

\begin{center}
\textbf{Задача 2.}
\end{center}

Обозначим случайную величину, распределение которой мы ищем, как $X$. Это дискретная случайная величина(она может принимать только значения из $\mathbb{Z}_+$, которых счетно). Найдем вероятность $\mathbb{P}(X = k)$.

Пусть $N$ -  количество экспирементов. Формула полной вероятности:

$$\mathbb{P}(X = k) = \mathbb{P}(N=j+r)$$

Для вычисления этих вероятностей вероятности будем пользоваться классическим определением.

$$\mathbb{P}(N=j) = \frac{C_j^r}{2^j}$$

$$\mathbb{P}(X=k) = \frac{C_{k+r}^r}{2^{k+r}}$$

Распределение случайной величины:

$$F_X(x) = \mathbb{P}(X<x) = \sum\limits_{k = 0}^{\lfloor x \rfloor}\mathbb{P}(X=k)$$

$$\boxed{\mathbb{P}(X=k) = \frac{C_{k+r}^r}{2^{k+r}}}$$

\begin{center}
\textbf{Задача 3.}
\end{center}

1. Найдем вероятность $\mathbb{P}(Z = k)$:

$$\mathbb{P}(Z = k) = \sum\limits_{j=0}^k\mathbb{P}(X = j)\mathbb{P}(Y = k-j)$$

$$\mathbb{P}(Z=k) = \sum\limits_{j=0}^k\frac{\lambda^j\mu^{k-j}}{j!(k-j)!}\exp(-(\lambda+\mu)) =$$
$$=\frac{1}{k!}\left(\sum\limits_{j=0}^kC_k^j\lambda^j\mu^{k-j}\right)e^{-(\lambda+\mu)}=$$
$$=\frac{(\mu+\lambda)^k}{k!}e^{-(\lambda+\mu)}$$

Распределение:

$$\boxed{\mathbb{P}(Z=k)=\frac{(\mu+\lambda)^k}{k!}e^{-(\lambda+\mu)}}$$

2. $$\mathbb{P}(X=k|Z=s)= \frac{\mathbb{P}(\{X=k\}\cap\{Z=s\})}{\mathbb{P}(Z=s)} = \frac{\mathbb{P}(Z=s|X=k)\mathbb{P}(X=k)}{\mathbb{P}(Z=s)}$$

Если $k \leq s$:

$$\mathbb{P}(Z=s|X=k) = \frac{\lambda^k\mu^{s-k}}{k!(s-k)!}\exp(-(\lambda+\mu)),$$
иначе эта вероятность равна нулю.

$$\mathbb{P}(X=k|Z=s)= \frac{\lambda^k\mu^{s-k}}{k!(s-k)!}e^{-(\lambda+\mu)} \frac{\lambda^k}{k!}e^{-\lambda}\left(\frac{(\mu+\lambda)^s}{s!}e^{-(\lambda+\mu)}\right)^{-1}$$

$$\mathbb{P}(X=k|Z=s) = \frac{C_s^k}{k!}\frac{\lambda^k\mu^{s-k}}{(\lambda + \mu)^s} e^{-\lambda}$$

Окончательное выражение для распределения:
\begin{equation*}
\boxed{\mathbb{P}(X=k|Z=s)=
 \begin{cases}
   \frac{C_s^k}{k!}\frac{\lambda^k\mu^{s-k}}{(\lambda + \mu)^s} e^{-\lambda}, &\text{if  $k\leq s$}\\
   0 &\text{if $k>s$}
 \end{cases}}
\end{equation*}
\end{document}