 \documentclass[12pt]{article}
\usepackage[T2A]{fontenc}
\usepackage[utf8]{inputenc}       

\usepackage[english,russian]{babel}
\usepackage{amsmath,amsfonts,amsthm,amssymb,amsbsy,amstext,amscd,amsxtra,multicol}
\usepackage{verbatim}
\usepackage{tikz}
\usetikzlibrary{automata,positioning}
\usepackage{multicol}
\usepackage{graphicx}
\usepackage[colorlinks,urlcolor=blue]{hyperref}
\usepackage[stable]{footmisc}
\usepackage{ dsfont }

\usepackage{xparse}
\usepackage{ifthen}
\usepackage{bm}
\usepackage{color}

\usepackage{algorithm}
\usepackage{algpseudocode}

\DeclareMathOperator*{\argmin}{argmin}
\DeclareMathOperator*{\argmax}{argmax}
\DeclareMathOperator*{\sign}{sign}
\newcommand\norm[1]{\left\lVert#1\right\rVert}

\begin{document}

\begin{center}
{Курузов Илья}

{Задание 5}
\end{center}

\begin{center}
\textbf{Задача 1.}
\end{center}

Пусть $\mathbb{I}_i^{test}$ и $\mathbb{I}_i^{stand}$- случайные величины, равные 1, если $i$-ый товар был отправлен на проверку или был стандартным соответственно, и равные нулю иначе. Пусть $N$ - количество бракованных изделий.

Интересующая нас вероятность:
$$P := \mathbb{P}(N = k|I_i^{test}\to I_i^{stand} = 1, \forall i) = $$
$$ =\sum\limits_{i_1\dots i_k \in \overline{1, n}, i_1\neq\dots i_n}  \mathbb{P}(\mathbb{I}_{i_j}^{stand} = 0, \forall j \in \overline{1, k}|I_i^{test}\to I_i^{stand} = 1, \forall i) \cdot $$
$$\cdot \mathbb{P}(\mathbb{I}_{m}^{stand} = 1, \forall m \neq i_j, j \in \overline{1, k}|I_i^{test}\to I_i^{stand} = 1, \forall i)$$  

Теперь плавно и постепенно найдем значение выше написанной вероятности.

$$\mathbb{P}(\mathbb{I}_{i_j}^{stand} = 0, \forall j \in \overline{1, k}|I_i^{test}\to I_i^{stand} = 1, \forall i) = $$
$$= \prod_{j = 1}^k \mathbb{P}(\mathbb{I}_{i_j}^{stand} = 0)\mathbb{P}(\mathbb{I}_{i_j}^{test} = 0)=$$
$$= \left((1-p)(1-q)\right)^k$$


$$\mathbb{P}(\mathbb{I}_{m}^{stand} = 1, \forall m \neq i_j, j \in \overline{1, k}|I_i^{test}\to I_i^{stand} = 1, \forall i) = $$
$$= \prod_{j = 1}^k \mathbb{P}(\mathbb{I}_{i_j}^{stand} = 1)= $$ 
$$ = p^k$$ 

Всего наборов $\{i_1\dots i_k \in \overline{1, n}, i_1\neq\dots i_n\}$ - $C_n^k$.

Искомая вероятность

$$\boxed{P = C_n^k \left(p(1-p)(1-q)\right)^k}$$
\begin{center}
\textbf{Задача 2.}
\end{center}

Наибольшее значение $p$ равно $\frac{2}{3}$.

1) Докажу его максимальность от противного. Допустим, что существуют  такие множества $A, B, C$, удовлетворяющие условию задачи, при чем для них $p >\frac{2}{3}$

Формула включений исключений:

$$\mathbb{P}(A \cup B\cup C) = 3p - \left(\mathbb{P}(A\cup B)+\mathbb{P}(C\cup B)+\mathbb{P}(A\cup C)\right) + \mathbb{P}(A\cup B \cup C)=$$
$$= 3p - \left(\mathbb{P}(A\cup B)+\mathbb{P}(C\cup B)+\mathbb{P}(A\cup C)\right) \geq$$
$$\geq 3p - \left((1 - \mathbb{P}(C)) + (1 - \mathbb{P}(A)) + (1 - \mathbb{P}(B))\right) = $$

$$=6p -1  > 1$$

Но с другой стороны $$\mathbb{P}(A \cup B\cup C) \geq \mathbb{P}(\Omega) =1$$. Противоречие, и, значит, наибольшее значение $p$ равно $\frac{2}{3}$.

2) Покажу, что наибольшее значение достигается.

Воспользуемся классическим определением вероятности. Множество элементарных исходов $\Omega=\{1, 2, 3\}$. 

$$A = \{1, 2\}, B = \{2, 3\}, C = \{1, 3\}$$
$$\mathbb{P}(A) = \mathbb{P}(B) = \mathbb{P}(C) = \frac{2}{3}$$
$$\mathbb{P}(A\cup B \cup C)= \mathbb{P}(\emptyset)=0$$

\begin{center}
\textbf{Задача 3.}
\end{center}

Распределение для $Y$


\begin{equation*}
F_Y(x) = \mathbb{P}(X^2 < x) = \mathbb{P}(X < \sqrt{x}) = 
 \begin{cases}
   1 &\text{if  $x > 1,$}\\
   \sqrt{x} &\text{if  $0 \leq x \leq 1,$}\\
   0 &\text{if $x < 0.$}
 \end{cases}
\end{equation*}


\begin{equation*}
\boxed{F_Y(x) =
 \begin{cases}
   1 &\text{if  $x > 1,$}\\
   \sqrt{x} &\text{if  $0 \leq x \leq 1,$}\\
   0 &\text{if $x < 0.$}
 \end{cases}}
\end{equation*}

\begin{center}
\textbf{Задача 4.}
\end{center}


\begin{equation*}
\mathbb{P}(Y = m) = \sum\limits_{n=1}^{+\infty}\mathbb{P}\left(\sum\limits_{i=1}^NX_i = m|N=n\right)\mathbb{P}(N=n) = \\
=\sum\limits_{n=1}^{+\infty}\mathbb{P}\left(\sum\limits_{i=1}^nX_i = m\right)\mathbb{P}(N=n)
\end{equation*}

\begin{equation*}
\mathbb{P}\left(\sum\limits_{i=1}^nX_i = m\right) = 
\begin{cases}
C_n^m p^m(1-p)^{n-m} &\text{if  $m \leq n,$}\\
0 &\text{if  $else$}\\
\end{cases}
\end{equation*}


\begin{equation*}
\mathbb{P}(N = n) = \frac{\lambda^n}{n!}e^{-\lambda}
\end{equation*}

С учетом выше написанного, получаю выражение для распределения:

\begin{equation*}
\begin{gathered}
\mathbb{P}(Y = m) = \sum\limits_{n=m}^{+\infty}C_n^m p^m(1-p)^{n-m}\frac{\lambda^n}{n!}e^{-\lambda} =\\=e^{-\lambda}\left(\frac{p}{1-p}\right)^m\sum\limits_{n=m}^{+\infty} C_n^m (1-p)^n\frac{\lambda^n}{n!} = \\
= e^{-\lambda}\left(\frac{p}{1-p}\right)^m\frac{1}{m!}\sum\limits_{n=m}^{+\infty} (1-p)^n\frac{\lambda^n}{(n-m)!} = \\
= e^{-\lambda}\frac{(p\lambda)^m}{m!}\sum\limits_{n=0}^{+\infty} (1-p)^n\frac{\lambda^n}{n!} = \\
= \frac{(p\lambda)^m}{m!}e^{-p\lambda}
\end{gathered}
\end{equation*}

Распределение для $Y$:

\begin{equation*}
\boxed{
\begin{gathered}
\mathbb{P}(Y = m) = \frac{(p\lambda)^m}{m!}e^{-p\lambda}\\
Y \sim Po(p\lambda)
\end{gathered}}
\end{equation*}

\begin{center}
\textbf{Задача 6.}
\end{center}

Пусть $\mathbb{I}_{a, b,c}$ -случайная величина, которая равна 1, если $a, b, c$ образуют треугольник в графе $G$ и 0 иначе.

$$\mathbb{P}(\mathbb{I}_{a, b,c} = 1) = p^3$$

Тогда искомое матожидание $N$ равно:

$$N = \mathbb{E}\left[\sum\limits_{a, b, c \in G}\mathbb{I}_{a, b,c}\right] =$$
$$= \sum\limits_{a, b, c \in G}\mathbb{E}\mathbb{I}_{a, b,c} = C_n^3 p^3$$

$$\boxed{N = p^3C_n^3}$$

\end{document}