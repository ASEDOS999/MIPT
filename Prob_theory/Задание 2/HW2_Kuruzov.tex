 \documentclass[12pt]{article}
\usepackage[T2A]{fontenc}
\usepackage[utf8]{inputenc}       

\usepackage[english,russian]{babel}
\usepackage{amsmath,amsfonts,amsthm,amssymb,amsbsy,amstext,amscd,amsxtra,multicol}
\usepackage{verbatim}
\usepackage{tikz}
\usetikzlibrary{automata,positioning}
\usepackage{multicol}
\usepackage{graphicx}
\usepackage[colorlinks,urlcolor=blue]{hyperref}
\usepackage[stable]{footmisc}
\usepackage{ dsfont }

\usepackage{xparse}
\usepackage{ifthen}
\usepackage{bm}
\usepackage{color}

\usepackage{algorithm}
\usepackage{algpseudocode}

\DeclareMathOperator*{\argmin}{argmin}
\DeclareMathOperator*{\argmax}{argmax}
\DeclareMathOperator*{\sign}{sign}
\newcommand\norm[1]{\left\lVert#1\right\rVert}

\begin{document}

\begin{center}
{Курузов Илья}

{Задание 2}
\end{center}

\begin{center}
\textbf{Задача 1.}
\end{center}

Пусть $B$ - событие, что за второй дверью коза, $A$ - событие, заключающеся в том, что за первой дверью авто. Требуется авто $\mathbb{P}(A|B)$.

$$\mathbb{P}(A|B)=\frac{\mathbb{P}(A\cap B)}{\mathbb{P}(B)}$$

Множество элементарных исходов $\Omega = $(А, К, К), (К, А, К), (К, К, А). $B = $(А, К, К), (К, К, А).$A =$(А, К, К).
$$\mathbb{P}(B)=\frac{2}{3}$$
$$\mathbb{P}(A\cap B)=\frac{1}{3}$$

$$\boxed{\mathbb{P}(A|B)=\frac{1}{2}}$$

\begin{center}
\textbf{Задача 2.}
\end{center}

Пусть $\pi_n$ - вероятность того, что число орлов после $n$ независимых подбрасываний будет четно.

Событие того, что число орлов после $n+1$ независимых подбрасываний будет четно, равно сумме событий: в предыдущих $n$ орлов четно и выпала решка или наоборот. Реккурентная формула:

$$\pi_{n+1} = (1-p)\pi_n+p(1-\pi_n)$$

$$\pi_{n+1} = (1-2p)\pi_n+p$$

Линейная реккурента

$$\pi_{n+1} - (1-2p)\pi_n = p$$

Характеристическое уравнение:

$$q -(1-2p)=0$$

$$q = 1-2p$$

$$\pi_n^0 = C(1-2p)^n$$

Найдем общее решение в виде $\pi_n=\pi_n^0+z$, где $z$ - неизвестная однородность:

$$z - (1 - 2p)z = p$$

$$z = \frac{1}{2}$$

Исходя из того, что $\pi_1 = 1$(выпала решка- выпало четное количество орлов), определяем константу $C$.

$$\boxed{\pi_n=(1-2p)^n+\frac{1}{2}}$$

\begin{center}
\textbf{Задача 4.}
\end{center}

1. Пусть $B_i$ - событие того, что выбрана $i$-ая урна. Тогда совокупность всех таких событий образует разбиение множества элементарных исходов. Пусть $A$ - событие того, что были достаны $k$ белых шаров и после этого 1 черный.

Пусть $w_i = \frac{i}{m}$, $b_i = \frac{m-i}{m}$ - вероятности того, что из $i$-ой урны достан белый или черный шар соответственно. Найдем $\mathbb{P}(A|B_i)$.

$$\mathbb{P}(A|B_i) = w_i^kb_i$$

$$\mathbb{P}(B_i) = \frac{1}{m+1}$$

Формула полной вероятности:

$$\mathbb{P}(A) = \sum\limits_{i=0}^{m}\mathbb{P}(A|B_i)\mathbb{P}(B_i)$$

$$\mathbb{P}(A) = \frac{1}{m+1}\sum\limits_{i=0}^{m} w_i^kb_i$$

$$\boxed{\mathbb{P}(A) = \frac{1}{m+1}\frac{1}{m^{k+1}}\sum\limits_{i=1}^{m-1} i^k(m-i)}$$

2. Сделаем небольшое преобразование полученной формулы:

$$\mathbb{P}(A) = \frac{1}{m+1}\frac{1}{m^{k+1}}\sum\limits_{i=1}^{m-1} i^k(m-i)$$

$$\mathbb{P}(A) = \frac{1}{m+1}\frac{1}{m^{k+1}}\left(m\sum\limits_{i=0}^{m} i^k-\sum\limits_{i=0}^{m} i^{k+1}\right)$$

$$\mathbb{P}(A) = \frac{1}{m+1}\left(\sum\limits_{i=0}^{m} \left(\frac{i}{m}\right)^k-\sum\limits_{i=0}^{m} \left(\frac{i}{m}\right)^{k+1}\right)$$

Пользуясь тем фактом, что $\sum\limits_{i=1}^ni^k = \frac{n^{k+1}}{k+1} + O(n^p)$, получаю:

$$\mathbb{P}(A) = \frac{1}{m+1}\left(\frac{m}{k+1}-\frac{m}{k+2}+O\left(\frac{1}{m}\right)\right)$$

$$\mathbb{P}(A) = \frac{m}{m+1}\left(\frac{1}{(k+1)(k+2)}+O\left(\frac{1}{m^2}\right)\right)$$

$$\boxed{\lim\limits_{m\rightarrow\infty}\mathbb{P}(A)=\frac{1}{(k+1)(k+2)}}$$


\begin{center}
\textbf{Задача 5.}
\end{center}

I. Интересующая нас вероятность $p=\mathbb{P}(\overline{\cup_{i=1}^nA_i})=\mathbb{P}(\cap_{i=1}^n\overline{A_i})$. Как было доказано в курсе, множество дополненений независимых в совокупности событий независимо в совокупности. 

$$\boxed{p= \prod\limits_{i = 1}^n(1-p_i)}$$

II. Из формулы Тейлора для $e^(-x)$ с остаточным членом в форме Лагранжа следует, что $$1 - p\leq\exp(-p)$$

Из этого, результата первого пункта и свойств экспоненты получаем $$\boxed{p\leq\exp\left(-\sum\limits_{i=1}^np_i\right)}$$

\end{document}