 \documentclass[12pt]{article}
\usepackage[T2A]{fontenc}
\usepackage[utf8]{inputenc}       

\usepackage[english,russian]{babel}
\usepackage{amsmath,amsfonts,amsthm,amssymb,amsbsy,amstext,amscd,amsxtra,multicol}
\usepackage{verbatim}
\usepackage{tikz}
\usetikzlibrary{automata,positioning}
\usepackage{multicol}
\usepackage{graphicx}
\usepackage[colorlinks,urlcolor=blue]{hyperref}
\usepackage[stable]{footmisc}
\usepackage{ dsfont }

\usepackage{xparse}
\usepackage{ifthen}
\usepackage{bm}
\usepackage{color}

\usepackage{algorithm}
\usepackage{algpseudocode}

\DeclareMathOperator*{\argmin}{argmin}
\DeclareMathOperator*{\argmax}{argmax}
\DeclareMathOperator*{\sign}{sign}
\newcommand\norm[1]{\left\lVert#1\right\rVert}

\begin{document}

\begin{center}
{Курузов Илья}
{Задание 1}
\end{center}

\begin{center}
\textbf{Задача 1.}
\end{center}

1. Множество элементарных исходов: $\Omega$ - все перестановки набора букв. 
$$|\Omega|=10!$$

2. Мощность интересующего нас множества событий (с учетом трех букв "А" и двух "Т") равна $$|A| = 3!*2!$$

3. $\mathbb{P}(A) = \frac{|A|}{|\Omega|} = \frac{12}{10!} $

$$\boxed{\mathbb{P}(A) = \frac{12}{10!} }$$

\begin{center}
\textbf{Задача 2.}
\end{center}

1. Множество всех элементарных исходов: $\Omega$ - множество $n$-мерных векторов($n$=23), в котором $i$-ая компонента - день рождения - $i$-ого ученика. $$|\Omega| = 365^{23}$$.

2. Событие $A$ - найдется две равные компоненты в случайном векторе. Событие $\Omega\backslash B$ - все дни рождение различны. $$|A| = |\Omega| - |\Omega\backslash B|= |\Omega| - C_{365}^{23}*23!$$

3. $$\boxed{\mathbb{P}(A) = 1 - \frac{365!}{342!365^{23}}}$$

\begin{center}
\textbf{Задача 3.}
\end{center}

I. Шары различимы.

1. Множество всех элементарных исходов - множество $n$-мерных векторов, у которых на месте каждой компоненты стоит номер ящика в котором находится данный шар.

$$|\Omega| = m^n$$

2. Интересующее нас подмножество - в компонентах вектора нет первого ящика.

$$|A|=(m-1)^n$$

$$\boxed{\mathbb{P}(A)=\left(\frac{m-1}{m}\right)^n}$$
II. Шары неразличимы.

1. Множество всех элементарных исходов - все вектора, сумма компонент каждого из которых равна $n$. В итоге получаем, что мощность множества равна количеству способов разбить число $n$ на $m$ упорядоченных целых неотрицательных слагаемых:

$$|\Omega| = {n+m-1 \choose m-1}$$

2. Интересующее нас подмножество - первая компонента равна нулю. Отсюда получаем $$|A| = {n+m-2 \choose m-2}$$

$$\boxed{\mathbb{P}(A)=\frac{m-1}{n+m-1}}$$


\begin{center}
\textbf{Задача 4.}
\end{center}

1. Множество всех элементарных исходов $\Omega = \{(A, B)|A, B \subset 2^N\}$. $$|\Omega|=2^{2N}$$

2. Множество благоприятных исходов - такие пары, пересечение элементов которых равно пустому множеству. Построим эти пары следующим способом: выберем $k$ элементов для $A$ и для $B$ будем выбирать из оставшихся. И так для всех $k$. Из этой идеи получаем мощность данного множества: 

$$|A| = \sum\limits_{j=0}^NC_n^j\left(\sum\limits_{i=0}^{N-j}C_{N-j}^i\right) = \sum\limits_{j=0}^NC_n^j 2^{N-j} = 3^N$$

$$\boxed{\mathbb{P}(A)=\left(\frac{3}{4}\right)^N}$$

\begin{center}
\textbf{Задача 5.}
\end{center}

1. Данное семейство не является $\sigma$-алгеброй. Докажу, что это семейство не замкнуто относительно счетного объединения.

Рассмотрим следующую числовую последовательность:

\begin{equation*}
b_n = 
 \begin{cases}
   2b_{n-1} &\text{if  $n \mod 2 = 0$}\\
   \frac{3}{2}b_{n-1} &\text{if $n \mod 2 \neq 0$}
 \end{cases}
\end{equation*}
$$b_1=2$$
Далее рассмотрим подпоследовательность множеств:

\begin{equation*}
A_n = 
 \begin{cases}
   \{1\ldots\frac{3}{4}b_n\} &\text{if  $n \mod 2 = 0$}\\
   \{1\ldots\frac{1}{2}b_n\} &\text{if $n \mod 2 \neq 0$}
 \end{cases}
\end{equation*}

Каждый из элементов последовательности принадлежит семейству(множество конечно, а, следовательно, предел для него равен нулю).

Покажем, что $A_{n-1}\cup A_n = A_{n+1}$. В случае, если $n \mod 2 \neq 0$ это очевидно, докажем другой случай.

$$A_{n-1}\cup A_n =  \{1\ldots\frac{3}{4}b_{n-1}\} \cup \{1\ldots\frac{1}{2}b_n\} = \{1\ldots\frac{3}{4}b_{n-1}\} \cup \{1\ldots\frac{3}{4}b_{n-1}\}=A_n$$

Что и требовалось доказать.

Рассмотрим числовую последовательность.

$$a_n = \frac{|\cup_{n=1}^{\infty} A_n \cap \{1\ldots b_n\}|}{b_n}$$

Видно, что для существования предела из условия для $\cup_{n=1}^{\infty} A_n$ должен существовать предел для выше описанной последовательности.

Однако, как видим

\begin{equation*}
a_n = 
 \begin{cases}
   \frac{3}{4} &\text{if  $n \mod 2 = 0$}\\
   \frac{1}{2} &\text{if $n \mod 2 \neq 0$}
 \end{cases}
\end{equation*}

Откуда следует, что не существует предела последовательности $a_n$, а, следовательно, и предела из условия для $\cup_{n=1}^{\infty} A_n$.

А, значит, данное семейство не является $\sigma$-алгеброй, поскольку это семейство не замкнуто относительно счетного объединения.

\end{document}